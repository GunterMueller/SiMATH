% \documentstyle[11pt]{book1}
% \setlength{\voffset}{-2cm}
% \oddsidemargin0.5cm   \evensidemargin0.5cm
% \parindent0pt
% \textwidth 6.0in \textheight 21.8cm
% \input texdefs.tex
% \begin{document}
% \pagestyle{plain}
% \setcounter{chapter}{1} % one less than the intended chapter number

\chapter{The Interface}

The SIMATH system uses a special {\em interface\/}\index{interface} to its libraries
permitting single-letter commands for editing, compiling, and printing of
SIMATH programs.

%%%%%%%%%%%%%%%%%%%%%%%%%%%%%%%%%%%%%%%
%%%%                               %%%%
%%%%       c h p 2 1 . t e x       %%%%
%%%%                               %%%%
%%%%%%%%%%%%%%%%%%%%%%%%%%%%%%%%%%%%%%%

\section{Starting SIMATH}
From your system shell type the command
\footnote{We use {\bf \%} as the standard prompt symbol.}
\leer
{\tt \% {\bf SM}\/  \care} \index{SM}
\leer
SIMATH's introductory screen will appear followed by
the five-line menu of SIMATH commands and the SIMATH prompt symbol $<$ on
the last line (see p.~\pageref{SMintro}). SIMATH is now ready to receive your command.


%%%%%%%%%%%%%%%%%%%%%%%%%%%%%%%%%%%%%%%
%%%%                               %%%%
%%%%       c h p 2 2 . t e x       %%%%
%%%%                               %%%%
%%%%%%%%%%%%%%%%%%%%%%%%%%%%%%%%%%%%%%%

\section{Example session}
We now give an example of a SIMATH session to show how the interface works.
For now, we shall restrict the examples to C programs. Later we will develop
actual SIMATH programs. We 
will create two C functions, {\bf gcd()} and {\bf lcm()}, then use these in a 
short program, the {\bf main()} function.

Note: SIMATH files are not necessarily located under {\bf /usr/local/simath}.
Any directory can be chosen upon installation, for example 
{\bf /usr/simath}. Here, we use \\
{\bf /usr/local/simath} as the default directory.
For this session, we assume that the file {\bf .SM.default} does not exist
(see \S 2.4.1).

\newpage

We now begin by executing SM
\leer
{\tt\% {\bf SM} \care} 
\label{SMintro}
\begin{verbatim}


                         **************************           
                         *                        *           
                         *      S I M A T H       *           
                         *                        *           
                         **************************           



                      Version 4.6 started at 11:34:17



 no source library
 no object library
 no file name

   (N)AME:     (e,E)dit     (p)re        (o)bj        (c)omp       (C)comp 
   object:     (a)r name    (d)ir        ar(+)obj     ar(-)obj     (r)anlib
  USR-SRC:     (A)R NAME    (D)IR        AR(*)SRC     AR(_)SRC     (S)ELECT
   SM-SRC:     E(X)TRACT    (~,@,.)                                        
          (!)  ($)  (?)     (P)RINT      (H)ELP       (R)UN        (Q)UIT  

<
\end{verbatim}

\leer
SIMATH commands are abbreviated and listed in a five-line menu.  For each command, it suffices
to type a single letter (given in parenthesis) followed by the RETURN key (except for \$).

Type {\bf N}\index{N} at the prompt to name the first C function to be created.
\leer
{\tt $<$ {\bf N} \care \\
file name:{\bf gcd} \care\\
$<$}
\leer
All input must be followed by RETURN. Typing errors can be corrected by
using the BACKSPACE and/or the DELETE key (depending on your machine).

\newpage

Invoke an editor (either with {\bf e} or with {\bf E}; see \S 2.4.1) and
type in the following program; it will automatically be saved  under
the name {\bf gcd.S}.
\leer
{\tt $<$ {\bf e} \care}\index{e}
\begin{verbatim}
#include <_simath.h>

int             gcd(a, b)
    int             a, b;
{
    int             c;
    
    if (a < b) 
      {c = a; a = b; b = c;}
    
    while (b != 0) 
      {c = a % b; a = b; b = c;}
    
    return (a);
}
\end{verbatim}
$<$   
\leer
Type {\bf H} to display the SIMATH menu (if necessary).
\leer
{\tt $<$ {\bf H} \care} \index{H}
\begin{verbatim}
 no source library
 no object library
 current file name:      gcd


   (N)AME:     (e,E)dit     (p)re        (o)bj        (c)omp       (C)comp 
   object:     (a)r name    (d)ir        ar(+)obj     ar(-)obj     (r)anlib
  USR-SRC:     (A)R NAME    (D)IR        AR(*)SRC     AR(_)SRC     (S)ELECT
   SM-SRC:     E(X)TRACT    (~,@,.)                                        
          (!)  ($)  (?)     (P)RINT      (H)ELP       (R)UN        (Q)UIT  
<
\end{verbatim}

We now want to create  an {\em object module library\/} which can archive
our compiled subroutines and make them available to the linker for later use.  
(Notice that all the libraries you create will be located in 
{\bf /usr/local/simath/lib.})
{\tt
\begin{tabbing}
1234567890123456\=23456123\=\kill
$<$ {\bf a} \care \index{a} \\
object library:\>/usr/local/simath/lib/lib$*$.a,\\
                  \>$*$ = {\bf example  \care }\\
$<$ \>   
\end{tabbing} }

The library libexample.a, in the directory {\bf /usr/local/simath/lib}, is now
the working library for SIMATH. We compile the program {\bf gcd.S} to
the object module {\bf gcd.o}. A {\em preprocessor\/} creates a temporary file
{\bf gcd.c} and passes it to the C compiler.

{\tt $<$ {\bf o}\index{o} \care}
\begin{verbatim}
 SM preprocessor started at  11:48:57

 SM preprocessor terminated correctly at  11:48:58

 Compiler started at  11:48:58

 Compiler terminated correctly at  11:49:03
\end{verbatim}

If the compiler reports an error, edit the program, using {\bf e} or {\bf E},
to make the necessary corrections.

We now add the object module gcd.o to the library.
\leer
{\tt $<$ {\bf +}\index{+} \care}
\begin{verbatim}
ar: creating /usr/local/simath/lib/libexample.a
 `ar r /usr/local/simath/lib/libexample.a gcd.o' done
\end{verbatim}
$<$
\leer
If the library {\bf libexample.a} already exists, only the line
\begin{verbatim}
 `ar r /usr/local/simath/lib/libexample.a gcd.o' done
\end{verbatim}
will appear on your screen. If the file gcd.o is already in
the library, you will get the following message on your screen:
\begin{verbatim}
SM: `gcd' already exists in `/usr/local/simath/lib/libexample.a'

 rw-r--r-- 431/42 164 Sep 2 09:13 1996 gcd.o

overwrite? (y/n)
\end{verbatim}
You must decide if you want to overwrite the existing file or not.

\newpage

Now use the commands {\bf N} (new file name: lcm), 
{\bf e, o, +} to create a second C function.

\begin{verbatim}
#include <_simath.h>

int             lcm(a, b)
    int             a, b;
{
    int             c;

    /* Result zero */
    if (a == 0 || b == 0)
        c = 0;
    else
    {
        /* General case */
        a = abs(a); b = abs(b); c = a / gcd(a, b);

        /* if result representable */
        if (~(1 << 31) / b >= c)
            c *= b;

        /* Replacement value if result too large */
        else
            c = -1;
    }
    return (c);
}
\end{verbatim}
If you need help, enter {\bf H}. The commands referring to your current module
library are listed on the second line of the menu. Before we turn to the main program,
we will run {\bf ranlib} and look at the contents of our object module library.

\label{NOTEranlib}
Note: On many machines it is not necessary to apply the {\bf ranlib} command to a library.
The command on the menu calls the corresponding (inoperative) command.
\leer
{\tt $<$ {\bf r}\index{r} \care}
\begin{verbatim}
`ranlib /usr/local/simath/lib/libexample.a' done
\end{verbatim}

\newpage

Let us look at the library contents.
\leer
{\tt $<$ {\bf d}\index{d} \care}
\begin{verbatim}
(rw-r--r--431/42     34 Sep 2 11:57 1996 __.SYMDEF)
 rw-r--r--431/42    164 Sep 2 11:51 1996 gcd.o
 rw-r--r--431/42    332 Sep 2 11:57 1996 lcm.o
\end{verbatim}
The file {\tt \_\_.SYMDEF} is the table of contents of our library created by
the {\bf ranlib} command.

We now come to the main program.
\leer
{\tt $<$ {\bf N} \care \\
current file name : lcm\\
new file  name \ \ \ \ : {\bf mainp}\\
$<$ {\bf e} \care }

\begin{verbatim}
#include <_simath.h>

main()
{
    int             a, b, c;
    char            as[11], bs[11];

    printf("\nEnter the numbers: \na = ");
    while (gets(as) == NULL)
        printf("Reading error \na = ");
    printf("b = ");
    while (gets(bs) == NULL)
        printf("Reading error \nb = ");

    /* converting the strings to numbers */
    a = atoi(as);
    b = atoi(bs);

    /* lcm */
    if ((c = lcm(a, b)) == -1)
        printf("Result is too large\n");
    else
        printf("lcm( %d, %d ) = %d\n", a, b, c);
}

\end{verbatim}

\newpage

Compile and link the program {\bf mainp.S} with the command
\leer
{\tt $<$ {\bf c}\index{c} \care }
\begin{verbatim}
 SM preprocessor started at  12:02:42

 SM preprocessor terminated correctly at  12:02:43

 Compiler started at  12:02:43

 Compiler terminated correctly at  12:03:03
\end{verbatim}
The linker uses the modules from the library {\bf libexample.a}. If the compiler
or the linker reports an error, edit the main program, using {\bf e} or {\bf E}, to
make the necessary corrections.
The {\bf executable program} is now stored under the name 
{\bf mainp.x}.  Type {\bf R} to run it. Here is an example.
\leer
{\tt
$<$ {\bf R}\index{R} \care \\
RUN "mainp" \\[1.0eM]
Enter the numbers:\\
a = {\bf 55 \care}\\
b = {\bf 22 \care}\\
lcm( 55, 22 ) = 110
}
\begin{verbatim} 
***   # GCs: 1        GC time: 0.00 s      # collected cells: 0        *** 
***   # blocks: 1     block size: 16383    # free cells:      16376    ***  
***   total CPU time: 0.03 s                                           ***   
\end{verbatim}
$<$
\leer
At the end of the execution, we automatically get information about memory
administration
\begin{itemize}
\item number of garbage collections ({\tt \# GCs}),
\item total time (in seconds) spent garbage collecting ({\tt GC time}),
\item total number of memory cells collected during garbage collection \\
      ({\tt \# collected cells}),
\item number of memory blocks used for the computation ({\tt \# blocks}),
\item size (in cells) of one block ({\tt block size}),
\item number of free memory cells ({\tt \# free cells}),
\end{itemize}
and the total computation time (in seconds) ({\tt total CPU time}), including
the time spent for memory administration.

A second test: type in {\bf R} once again.
\leer
{\tt
$<$ {\bf R} \care \\
RUN "mainp"\\[1.0eM]
Enter the numbers:\\
a = {\bf 44444 \care}\\
b = {\bf 10000000 \care}\\
Result is too large
}
\begin{verbatim}
***   # GCs: 1        GC time: 0.00 s      # collected cells: 0        *** 
***   # blocks: 1     block size: 16383    # free cells:      16376    ***  
***   total CPU time: 0.03 s                                           ***   
\end{verbatim}
$<$
\leer
The result exceeds the maximum representable number in C, $2^{31}-1$.
Enter {\bf Q} to end the SIMATH session.
\leer
{\tt $<$ {\bf Q}\index{Q} \care}
\begin{verbatim}


                SM terminated at 12:06:47 

%
\end{verbatim}

\newpage


%%%%%%%%%%%%%%%%%%%%%%%%%%%%%%%%%%%%%%%
%%%%                               %%%%
%%%%       c h p 2 3 . t e x       %%%%
%%%%                               %%%%
%%%%%%%%%%%%%%%%%%%%%%%%%%%%%%%%%%%%%%%

\section{Description of SIMATH commands}

\subsection{Overview}
The SIMATH commands are listed in a five-line menu.  Execution of those
commands necessitates only typing the symbol in parenthesis followed by the RETURN
key (except for {\bf \$}). All input is done in {\em buffered\/} mode, i.e. typing
errors can be corrected by using the BACKSPACE and/or DELETE key.

Whenever a SIMATH command requires the input of a file (or library) name, the last
name given as input for that SIMATH command will be shown on your screen as the
current file (or library) name (see \S 2.3, {\bf mainp} function example). You must
either hit the RETURN key to keep the current name or enter a new name (in which
case the current name will be overwritten by the new name).

Note: At the beginning of a SIMATH session, the names of the source library, object 
library, and program defined by the last {\bf A}, {\bf a}, and {\bf N} commands of
the previous session will automatically be read in from the file {\bf .SM.default};
they are shown on your screen in the introductory lines.  If the file {\bf .SM.default}
does not exist or if there is no preinstallation, the following lines will appear on 
your screen
\leer
{\tt
no source library\\  
no object library\\
no file name
}
\leer
At the end of a SIMATH session, the names last preinstalled with the {\bf A}, {\bf a},
and {\bf N} commands will be written into the file {\bf .SM.default} in the directory
from which SIMATH was started (if you have writing permission).

Next is a short description of the SIMATH commands. A detailed description of each command
will be given in the following sections.

\newpage
\begin{tabular}{lcl}\index{commands}
Command         & Input          & \ Description \\
\cline{1-3}\\
NAME            & N              & \ enter program name \\
edit            & e              & \ start editor (see note below) \\
Edit            & E              & \ start editor (see note below) \\
pre             & p              & \ start preprocessor \\
obj             & o              & \ create object module \\
comp            & c              & \ compile and link executable program \\
Comp            & C              & \ compile and link executable program in the background \\
ar name         & a              & \ enter module library name \\
dir             & d              & \ list contents of the module library \\
ar$+$obj        & $+$            & \ archive object module into the module library\\
ar$-$obj        & $-$            & \ remove object module from the module library \\
ranlib          & r              & \ supply module library with table of contents \\
AR NAME         & A              & \ enter source library name \\
DIR             & D              & \ list contents of the source library \\
AR$*$SRC        & $*$            & \ archive source file into the source library\\
AR$\_$SRC       & $\_$           & \ remove source file from the source library \\
EXTRACT         & X              & \ extracts the sources of a SIMATH-function
to the current \\  
                &                & \ working directory. The function-name must be defined by
'N'. \\
SELECT          & S              & \ copy source file from the usr-source library to the
                                    current \\
                &               & \ working directory \\
!               & !               & \ branch into a subshell \\
\$              & \$              & \ execute a SHELL command \\
?               & ?               & \ look up keywords \\
$\tilde{\ }$, @ & $\tilde{\ }$, @ & \ look up on-line documentation \\
.               & .               & \ look up short description \\
PRINT           & P               & \ print source file \\
HELP            & H               & \ display menu \\
RUN             & R               & \ run program \\
QUIT            & Q               & \ end SIMATH session
\end{tabular}
\leer
\leer
Note: {\bf e} usually calls vi and {\bf E} emacs, but this depends on your
installation (see installation instructions).

\newpage

\subsection{Menu-line 1:\ \ Editing and compiling programs}                      
Note: ``similar shell commands'' refer to standard UNIX commands; they may differ
from your machine's shell commands.
\leer
\menurowone
\begin{tabular}{p{1.1in}p{4.4in}}
(N)AME:  & Enter program name\\
         & \\ 
Key      & $<$ {\bf N} \care\index{N}\\
         & file name: {\bf pname} \care\\   
         & $<$\\
         & \\ 
or       & $<$ {\bf N} \care\\
         & current file name: pnameold\\
         & new file name:\ \ \ \ \ \,{\bf pname} \care\\
         & $<$\\ 
         & \\ 
Function & {\bf pname} is the new preinstalled program, i.e. the new current
           program to be worked on. If only the {\mbox RETURN} key is hit,
           the current file {\bf pnameold} remains valid.\\
         & \\ 
Remark   & Suffixes are used internally to differentiate between
           the SM source \\
         & file {\bf pname.S}, the SM preprocessor versions {\bf pname.P} and \\
         & {\bf pname.c},
           the C preprocessor version {\bf pname.i}, 
           the object module {\bf pname.o}, the list of compiler errors {\bf pname$\_$CC}
           (when compiling in the background), and the executable program {\bf pname.x}.
\end{tabular}

\menurowone
\begin{tabular}{p{1.1in}p{4.4in}}
(e)dit        & Start editor\\
              & \\ 
Key           & $<$ {\bf e} \care\index{e}\\
              & \\ 
Function      & The preinstalled file is edited using the editor corresponding to
                {\bf e}, usually vi (see note \S 2.4.1).\\
              & \\
similar       & \\
shell command & {\bf \% vi pname.S } \care
\end{tabular}

\newpage

\menurowone
\begin{tabular}{p{1.1in}p{4.4in}}
(E)dit        & Start editor\\
              & \\
Key           & $<$ {\bf E} \care\index{E}\\
              & \\    
Function      & The preinstalled file is edited using the editor corresponding to
                {\bf E}, usually emacs (see note \S 2.4.1).\\
              & \\
similar       & \\
shell command & {\bf \% emacs pname.S } \care
\end{tabular}

\menurowone
\begin{tabular}{p{1.1in}p{4.4in}}
(p)re         & Start preprocessor\\
              & \\ 
Key           & $<$ {\bf p} \care\index{p}\\
              & \\ 
Function      & The SIMATH preprocessors and C preprocessor are run with the preinstalled file {\bf pname.S}
                as input. The SIMATH preprocessor creates the temporary files {\bf pname.P} and
                {\bf pname.c}. {\bf pname.c}, the C program equivalent of {\bf pname.S}, is then
                passed to the C preprocessor which creates the file {\bf pname.i}.\\
               & \\ 
similar        & \\
shell command  & {\bf \% cc -P -I. -I/usr/local/simath/include pname.c \care}\\
               & \\ 
Remark         & {\bf (p)re} is meant to be used mainly as a debugging tool.
\end{tabular}

\menurowone
\begin{tabular}{p{1.1in}p{4.4in}} 
(o)bj         & Create object module\\
              & \\  
Key           & $<$ {\bf o} \care\index{o}\\
              & \\ 
Function      & The C compiler compiles the file {\bf pname.c} to the object module 
                {\bf pname.o} after the SM preprocessors have been run automatically. Any compiling
                error will appear on your screen; use the {\bf e} or {\bf E} command to
                correct the error(s) in {\bf pname.S}.\\
              & \\ 
similar       & \\
shell command & {\bf \% cc -O -I. -I/usr/local/simath/include -c pname.c \care}
\end{tabular}

\newpage

\menurowone
\begin{tabular}{p{1.1in}p{4.4in}} 
(c)omp        & Compile and link an executable program\\
              & \\ 
Key           & $<$ {\bf c} \care\index{c}\\
              & \\
Function      & The C compiler creates an executable program {\bf pname.x} from 
                {\bf pname.S} where {\bf pname.S} must contain exactly one 
                main function.  The SM preprocessors are run automatically. Required
                modules from C and/or SIMATH libraries are linked automatically;
                any preinstalled, user-defined module library (see command \\
              & {\bf (a)r name})
                has priority over the system libraries. Any compiling or linking error will
                be shown on your screen; use the {\bf e} or {\bf E} command to correct
                them.\\
              & \\ 
similar       & {\bf \% cc  -O -I. -I/usr/local/simath/include} \\
shell command & \ \ \ {\bf -L/usr/local/simath/lib [-lexample] -lkern -lec4 \ldots}\\
              & \ \ \ {\bf -llist -lm -ltermcap -s -o pname.x pname.c \care}\\
              & \\
Remark        & see remark for (C)omp
\end{tabular}

\newpage

\menurowone
\begin{tabular}{p{1.1in}p{4.4in}} 
(C)omp   & Compile in the background\\
         & \\ 
Key      & $<$ {\bf C} \care\index{C}\\
         & Confirm with RETURN or enter additional module libraries
           {\bf ar1 ar2 \ldots}; {\bf ar1} will be linked first followed
           by {\bf ar2 \ldots}\ .\\
         & {\bf : [ar1 \ldots]\ \ \care}\\
         & \\
Function & This is the same function as {\bf (c)omp} except that everything
           runs in the background. Optional, additional module libraries
           {\bf ar1 \ldots} have priority over system libraries during linking
           ({\bf ar$*$} stands for {\bf /usr/local/simath/lib/libar$*$.a}); uppermost
           priority is still given to the preinstalled, user-defined module library
           (see command {\bf (a)r name}).\newline
           Compiling or linking errors will be listed in the file {\bf pname\_CC};
           use the {\bf e} or {\bf E} command to correct them.\\
         & \\
Remark   & In order to save space, the {\bf c} and {\bf C} commands
           automatically execute a {\bf strip} command (i.e the linker {\bf ld} is run
           with the -s option) which removes the symbol table and relocation bits
           ordinarily attached to the output of the assembler and linker. If you want
           to use a debugger, use the corresponding shell commands\\
         & {\bf c:\ \ \ \% CCC pname [ar1 ar2 \ldots] \ \ \ \ \ \ \care}\\
         & {\bf C:\ \ \% CCC pname [ar1 ar2 \ldots] \& \ \ \ \care}\\
         & where {\bf ar$*$}  stands for {\bf /usr/local/simath/lib/libar$*$.a}.
           The libraries {\bf ar1}, {\bf ar2} \ldots will take precedence over
           the system libraries during linking.
\end{tabular}

\newpage
 
\subsection{Menu-line 2:\ \  Archiving object modules}

\menurowtwo
\begin{tabular}{p{1.1in}p{4.4in}} 
(a)r name & Enter module library name\\
          & \\ 
Key       & $<$ {\bf a} \care\index{a}\\
          & object library: /usr/local/simath/lib/lib$*$.a,\\
          & \hspace*{75pt} $*$ = {\bf lname} \care\\ 
          & $<$\\
          & \\
or        & $<$ {\bf a} \care\\
          & current object library: /usr/local/simath/lib/liblnameold.a\\
          & new object library: \ \ \ \ /usr/local/simath/lib/lib$*$.a ,\\
          & \hspace*{113pt} $*$ = {\bf lname} \care\\
          & $<$ \\  
          & \\ 
Function  & {\bf liblname.a} is the new preinstalled object module library, i.e.
            your own object module library to be used with highest priority by
            the linker; it is located in the directory {\bf /usr/local/simath/lib}. 
            If only the RETURN key is hit, the current object library {\bf liblnameold.a}
            will remain valid.\\
          & \\
Remark    & If you no longer need your preinstalled module library, type in
            {\bf .}  for {\bf lname} to cancel the preinstallation.
\end{tabular}

\menurowtwo
\begin{tabular}{p{1.1in}p{4.4in}}
(d)ir         & List contents of the module library\\
              & \\ 
Key           & $<$ {\bf d} \care\index{d}\\
              & \\ 
Function      & The files contained in the preinstalled object module library 
                are listed; additional information such as file attributes, file size,
                and date of creation are also given.\\
              & \\
similar       & \\
shell command & {\bf \% ar tv /usr/local/simath/lib/liblname.a $|$ page -d \care}\\
              & \\
Remark        & If there is no preinstalled library or if the library does not
                exist yet, an error message will appear on your screen.
\end{tabular}

\newpage

\menurowtwo
\begin{tabular}{p{1.1in}p{4.4in}}
ar(+)obj       & Archive object module\\
               & \\ 
Key            & $<$ {\bf +} \care\index{+}\\
               & \\ 
Function       & The object module {\bf pname.o}, created from the preinstalled
                 pro\-gram {\bf pname.S} by the command {\bf o}, is added
                 to the preinstalled object module library and deleted from the
                 current working directory.\\
               & If the preinstalled library does not exist, it will be created.\\ 
               & If {\bf pname.o} already exists in the library, the user will be asked
                 whether or not to replace the {\bf pname.o} already in the library 
                 by the {\bf pname.o} from the current working directory.\\
               & If {\bf pname.o} does not exists, an error message will appear on your
                 screen.\\
               & \\
       similar & {\bf \% ar r /usr/local/simath/lib/liblname.a pname.o \care}\\
shell commands & {\bf \% rm pname.o  \care}\\ 
               & \\
Remarks        & If no library or program name is preinstalled or if an error occurs during
                 archiving, an error message will be displayed on your screen and the module
                 {\bf pname.o} will not be deleted from the current working directory.\\
               & After successful archiving, the following message will be displayed on
                 your screen:\\
               & `ar r /usr/local/simath/lib/liblname.a pname.o' done\\
\end{tabular}

\newpage

\menurowtwo
\begin{tabular}{p{1.1in}p{4.4in}}
 ar($-$)obj    & Remove object module from the library\\
               & \\ 
Key            &  $<$ {\bf $-$} \care\index{-}\\
               & \\ 
Function       & The object module {\bf pname.o}, created from the preinstalled
                 pro\-gram {\bf pname.S} by the command {\bf o}, will be 
	 	 removed from the preinstalled object module library.\\
               & If {\bf pname.o} is not in the library, an error message will
                 appear on your screen.\\
               & \\
   similar     & \\
shell command  & {\bf \% ar d /usr/local/simath/lib/liblname.a pname.o \care}\\   
               & \\ 
Remarks        & If no library or program name is preinstalled or if an error occurs,
                 an error message will be displayed on your screen.\\
               & After successful execution, the following message will be displayed
                 on your screen:\\
               & `pname' deleted from `/usr/local/simath/lib/liblname.a'
\end{tabular}
\leer
\menurowtwo
\begin{tabular}{p{1.1in}p{4.4in}}
(r)anlib       & Supply module library with table of contents\\
               & \\ 
Key            & $<$ {\bf r} \care\index{r}\\
               & \\ 
Function       & The preinstalled object module library will be supplied with a
                 table of contents \_\_.SYMDEF. On many workstations, {\bf ranlib}
                 is not necessary; the command {\bf r} has no effect (see note
                 p.~\pageref{NOTEranlib}).\\
similar        & \\
shell command  & {\bf \% ranlib /usr/local/simath/lib/liblname.a \care}\\    
               & \\ 
Remarks        & If {\bf ranlib} is necessary on your machine, then
                 you should run {\bf ranlib} again after any change to the
                 object library in order to keep the table of contents for
                 archive up to date; otherwise, you will get a warning from the
                 linker when you use the {\bf c} or {\bf C} command.\\
               & If no library is preinstalled or if an error occurs, an error message
                 will be displayed on your screen.\\
\end{tabular}
  
\newpage

\subsection{Menu-line 3:\ \   Archiving source files}
Lines 2 and 3 in the menu contain similar commands referring to object module
libraries and source libraries respectively; the only difference is that you
also have to press the Shift key for the commands in line 3.

These SIMATH commands are meant to help you organize your program source files 
into some meaningful order. You do not have to use them.
\leer
\menurowthree
\begin{tabular}{p{1.1in}p{4.4in}}   
(A)R NAME      & Enter source library name\\ 
               & \\ 
Key            & $<$ {\bf A} \care\index{A}\\
               & source library: {\bf USR-SNAME \care}\\   
               & $<$\\
               & \\ 
       or      & $<$ {\bf A} \care\\
               & current source library: USR-SNAMEOLD\\
               & new source library: \ \ \ \ {\bf USR-SNAME \care}\\ 
               & \\ 
Function       & {\bf USR-SNAME} is the new preinstalled source library, i.e. a library
                 where you may archive your program source files; it is located in the
                 working directory. If only the RETURN key is hit, the current source library
                 {\bf USR-SNAMEOLD} will remain valid.\\
               & \\  
\end{tabular}

\newpage

\menurowthree
\begin{tabular}{p{1.1in}p{4.4in}} 
(D)IR          & List contents of the source library\\
               & \\ 
Key            & $<$ {\bf D} \care\index{D}\\
               & \\ 
Function       & The files contained in the preinstalled source library are listed;
                 additional information such as file attributes, file size, and date
                 of creation are also given.\\
               & \\ 
    similar    & \\
 shell command & {\bf \% ar tv USR-SNAME $|$ page -d \care}\\
               & \\ 
Remark         & If there is no preinstalled library or if the library does not
                 exist yet, an error message will appear on your screen.
\end{tabular}

\menurowthree
\begin{tabular}{p{1.1in}p{4.4in}} 
AR($*$)SRC     & Archive source file\\
               & \\ 
Key            & $<$ {\bf $*$} \care\index{*}\\ 
               & \\ 
Function       & The preinstalled file {\bf pname.S} is added to the preinstalled source
                 library and deleted from the current working directory.\\
               & If the preinstalled library does not exist, it will be created.\\
               & If {\bf pname.S} already exists in the library, the user will be asked
                 whether or not to replace the {\bf pname.S} already in the library by
                 the {\bf pname.S} from the current working directory.\\
               & If {\bf pname.S} does not exist, an error message will appear on your
                 screen.\\
               & \\ 
       similar & {\bf \% ar r USR-SNAME pname.S \care}\\   
shell commands & {\bf \% rm pname.S \care}\\    
               & \\
Remarks        & If no library or program name is preinstalled or if an error occurs during
                 archiving, an error message will be displayed on your screen and the file
                 {\bf pname.S} will not be deleted from the current working directory. \\
               & After successful archiving, the following message will be displayed on
                 your screen:\\
               & `ar r USR-SNAME pname.S' done\\
\end{tabular}

\newpage

\menurowthree
\begin{tabular}{p{1.1in}p{4.4in}} 
AR($\_$)SRC    & Remove source file from the library\\
               & \\ 
Key            & $<$ {\bf $\_$} \care\index{_}\\ 
               & \\ 
Function       & The preinstalled source file {\bf pname.S} will be removed from the
                 preinstalled source library.\\
               & If {\bf pname.S} is not in the library, an error message will appear
                 on your screen.\\
               & \\   
    similar    & \\
 shell command & {\bf \% ar d USR-SNAME pname.S \care}\\    
               & \\ 
Remarks        & If no library or program name is preinstalled or if an error occurs,
                 an error message will be displayed on your screen.\\
               & After successful execution, the following message will be displayed
                 on your screen:\\
               & `pname'\ deleted from `USR-SNAME'  
\end{tabular}

\menurowthree
\begin{tabular}{p{1.1in}p{4.4in}}  
SELECT         & Copy source file from the source library to the current directory\\
               & \\ 
Key            & $<$ {\bf S} \care\index{S}\\
               & \\ 
Function       & The preinstalled file  {\bf pname.S} will be copied from the preinstalled
                 source library to the current working directory; {\bf pname.S} is not
                 deleted from the library.\\
               & If {\bf pname.S} already exists in the current directory, the user will
                 be asked whether or not to replace the {\bf pname.S} already in the
                 current directory by the {\bf pname.S} from the source library.\\
               & If {\bf pname.S} is not in the library, an error message will appear
                 on your screen.\\
               & \\ 
   similar     & \\
 shell command & {\bf \% ar x USR-SNAME pname.S  \care}\\  
               & \\
Remarks        & If no library or program name is preinstalled or if an error occurs,
                 an error message will be displayed on your screen.\\
               & After successful execution, the following message will be displayed
                 on your screen:\\
               & `pname.S' extracted from `USR-SNAME'
\end{tabular}
  
\newpage


\subsection{Menu-line 4:\ \ SIMATH-sources commands}
\menurowfour
\begin{tabular}{p{1.1in}p{4.4in}}
E(X)TRACT       & Extracts a SIMATH-function.\\
                & \\
Key             & $<$ {\bf X} \index{X} \\
                & \\
Function        & If you have defined a Name of a SIMATH-function with \\
                & the command 'N', X will try to find this function in \\
                & the SIMATH-source directory and copies it into the \\
                & current working directory.
\end{tabular}
%
\menurowfour
\begin{tabular}{p{1.1in}p{4.4in}}
($\tilde{\ }$,@,\,.)   & Look up on-line documentation\\
                       & \\
Key                    & $<$ {\bf $\tilde{\ }$} \index{$\tilde$}\care \ \ \ \ \ or \ \ \ \ \ $<$ {\bf @}
                         \care\\
                       & documentation of: {\bf xyz \care}\\
                       & \\
                       & $<$ {\bf .} \index{.}\care\\
                       & short description of: {\bf xyz \care}\\
                       & \\
Function               & In the first two cases a detailed description of the
                         SIMATH function {\bf xyz} is displayed. In the third
                         case, the description is displayed in TeX format, if
                         the SIMATH function does support a TeX documentation.
                         If the description is longer than one page, type
                         {\bf SPACE} to continue or {\bf q} to quit.

                         If you use the command \ {\bf .}\ , the short
                         description of the SIMATH function {\bf xyz} is
                         displayed, i.e. the function name with parameters and
                         the detailed function name.


                         Note: the search is case sensitive, e.g {\bf Xyz} would
                         not be found.\\
                       & \\
Remark                 & The tilde $\tilde{\ }$ serves as an escape character when a
                         remote login, using {\bf rlogin}, is performed. In this
                         case, SM might not accept $\tilde{\ }$; you should use
                         @ instead.
\end{tabular}

\newpage

\subsection{Menu-line 5:\ \ General commands}
\menurowfive
\begin{tabular}{p{1.1in}p{4.4in}}           
(!)            & Branch into a subshell\\
               & \\ 
Key            & $<$ {\bf !} \index{!}\care\\
               & \\ 
Function       & Branch into a subshell, i.e. the shell program specified in your
                 environment variable SHELL is invoked. You can work in the subshell
                 without losing any information preinstalled in SM. To leave the
                 subshell, type 'exit' or 'CTRL/D'.
\end{tabular}

\menurowfive
\begin{tabular}{p{1.1in}p{4.4in}}           
(\$)           & Execute a SHELL command\\
               & \\ 
Key            & $<$ {\bf \$ command \care}\index{$}\\    
               & \\ 
Function       & The shell command {\bf command} is executed; no need to
                 branch into a subshell using {\bf !}.\\
               & \\
Remark         & The {\bf cd} command does not change your working directory.
\end{tabular}

\menurowfive
\begin{tabular}{p{1.1in}p{4.4in}}           
(?)            & Look up keywords\\
               & \\ 
Key            & $<$ {\bf ?} \index{?}\care\\
               & \\ 
Function       & With the SM keyword index, you can search for SIMATH functions
                 which match (or match not) your keywords. 

                 After entering the SM keyword index, type {\bf ?} for help.
                 A key is an {\em egrep regular expression}. You can add a key 
                 to the list ({\bf +}), remove a key ({\bf -}), change a key 
                 ({\bf c/C}), or delete the entire list ({\bf 0}). The number 
                 of keys is limited by 9. The list is printed by the command
                 {\bf l/L}. 

                 With the commands {\bf p/P, @/$\tilde{\ }$, s/S}
                 you can print the matching SIMATH functions, print the
                 documentation of a function or print the source code of a
                 function, respectively. You can quit the keyword index to
                 SM with {\bf q/Q}.
                
\end{tabular}

\newpage

\menurowfive
\begin{tabular}{p{1.1in}p{4.4in}}           
(P)RINT        & Print source file\\
               & \\ 
Key            & $<$ {\bf P} \index{P}\care\\
               & \\
Function       & {\bf /usr/local/simath/bin/dr} prints the file {\bf pname.S}.\\
               & (By default, {\bf /usr/local/simath/bin/dr} contains two lines\\[1.5ex]
               & \hspace*{2cm}\#!/bin/sh\\
               & \hspace*{2cm}lpr \$ $*$\\[1.5ex]
               & You should change it according to your needs.)\\
               & \\
 similar       & \\
 shell command & {\bf \% /usr/local/simath/bin/dr pname.S\ \ \care}
\end{tabular}
\leer
\menurowfive
\begin{tabular}{p{1.1in}p{4.4in}}           
(H)ELP         & Display menu\\
               & \\ 
Key            & $<$ {\bf H} \index{H}\care\\
               & \\ 
Function       & The menu of the SIMATH commands is displayed.
\end{tabular}

\menurowfive
\begin{tabular}{p{1.1in}p{4.4in}}           
(R)UN          & Run program\\
               & \\ 
Key            & $<$ {\bf R} \index{R} \care\\
               & \\ 
Function       & The executable program {\bf pname.x}, created from the preinstalled program
                 {\bf pname.S} by the command {\bf c} or {\bf C} will be executed.\\
               & \\
   similar     & \\
 shell command & {\bf \% pname.x \care}
\end{tabular}
\leer
\menurowfive
\begin{tabular}{p{1.1in}p{4.4in}}            
(Q)UIT         & End SIMATH session\\
               & \\ 
Key            & $<$ {\bf Q} \index{Q}\care\\
               & \\ 
Function       & SM is terminated.
\end {tabular}
\leer 
\leer 
\leer 
{Note}: In SM, you can enter several commands on one line, e.g.

{\tt $<$ {\bf o + r} \care}

instead of

{\tt
 $<$ {\bf o}\ \ \care\\
 $<$ {\bf +} \care\\
 $<$ {\bf r}\ \ \care} 

\noindent
Trial and error will tell you what can be entered on a single line !

% \end{document}
