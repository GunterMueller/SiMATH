% \documentstyle[11pt]{book1}
% \setlength{\voffset}{-2cm}
% \oddsidemargin0.5cm   \evensidemargin0.5cm
% \parindent0pt
% \textwidth 6.0in \textheight 21.8cm
% \input texdefs.tex
% \begin{document}
% \pagestyle{plain}
% \setcounter{chapter}{0} % one less than the intended chapter number

\chapter{Introduction to  SIMATH}


%%%%%%%%%%%%%%%%%%%%%%%%%%%%%%%%%%%%%%%%
%%%%                               %%%%
%%%%       c h p 1 1 . t e x       %%%%
%%%%                               %%%%
%%%%%%%%%%%%%%%%%%%%%%%%%%%%%%%%%%%%%%%

\section{Introducing SIMATH}\index{SIMATH} 
SIMATH, that is {\em SInix--MATHematik\/}, is a computer algebra 
system focusing mainly on {\em algebraic number theory\/}.  
It is being developed at Tokyo Metropolitan University (TMU), Japan.
System development started on SINIX PCs and
up to now, we are running SIMATH on
\begin{itemize}
\item{} HP 9000 series 700 under HP-UX 9.0x and HP-UX 10.x
\item{} SGI machines under IRIX 5.3
\item{} Sun SPARCstation under SunOS 4.1.1
\item{} PCs under Linux 1.x and 2.x
\item{} hp AlphaServer under Tru64 UNIX
\end{itemize}
It should not be difficult to compile and run SIMATH on most UNIX
platforms. For example, SIMATH is known to run on SPARCstations under
Solaris 2.x and Intel based PCs under FreeBSD 4.x

\newpage

{\bf How does SIMATH differ from other systems ?}\index{SIMATH}  

The difference between SIMATH and most other systems lies essentially in 
\begin{itemize}
\item the main area of application: {\em algebraic number theory\/}.
\item the concept of the system: SIMATH is {\em a transparent system\/} which means that all sources are part of the system so the user can adapt existing general algorithms
to specific problems and integrate her/his own algorithms at any point within the system.
\item the programming language: SIMATH is written in C; likewise, the user works in C.
      SIMATH functions are integrated into a C-program simply by function calls.
\end{itemize}
SIMATH may also be accessed via the interactive calculator {\em simcalc\/} which features
\begin{itemize}
\item many of the existing SIMATH algorithms,
\item comprehensive error checking,
\item detailed ``help facilities''. 
\end{itemize}
This makes {\em simcalc\/} particularly suitable for a quick calculation on the side and
users with little programming experience.

  We have set up a mailing list for the users of the number theoretic 
system SIMATH. 
If you want to subscribe to this list send e-mail to
\begin{center}
simath-user-request@tnt.math.metro-u.ac.jp
\end{center}
and include the line 
\begin{center}
subscribe
\end{center}
in the body of your mail. The purpose of the list is to announce new versions 
of SIMATH and to discuss anything which is related to SIMATH and of public 
interest.

  The latest version of SIMATH may be obtained by anonymous ftp from
simath.math.metro-u.ac.jp (133.86.76.12) in the directory /simath/latest. 
For more information about the installation of SIMATH see \S 1.3.

If you have any questions or problems, please contact the following address:
\begin{center}
\begin{tabular}{ll}
SIMATH development Group & \\
Prof. Dr. Ken Nakamula &   \\
Department of Mathematics, & \\
Tokyo Metropolitan University & \\
Postcode 192-0397 & \\
Minami-Osawa 1-1, Hachioji-shi Tokyo, JAPAN & \\
& \\
e-mail: mail@simath.info &\\
Phone: +81-(0)426-77-2471
\end{tabular}
\end{center}  
\newpage


%%%%%%%%%%%%%%%%%%%%%%%%%%%%%%%%%%%%%%%
%%%%                               %%%%
%%%%       c h p 1 2 . t e x       %%%%
%%%%                               %%%%
%%%%%%%%%%%%%%%%%%%%%%%%%%%%%%%%%%%%%%%

\section{The logical structure of SIMATH}\index{SIMATH}

The following diagram illustrates the structure of the system.

\begin{center}
\begin{tabular}{|c|c|c|}
\hline
& & \\
\fbox{\fbox{simcalc}} & \fbox{\fbox{
\begin{tabular}{c}
       number theory \\
       package
       \end{tabular}}} & \fbox{\fbox{
\begin{tabular}{c}
        user \\
        applications    
        \end{tabular}}} \\
& & \\
\hline
\multicolumn{3}{c}{~} \\
\hline
& & \\
& \fbox{\B\quad elliptic curves\quad\B} & \\ 
& & \\ \cline {2-2} 
& & \\
& \fbox{\begin{tabular}{c}
            algebraic \\        
            number \\    
            fields 
        \end{tabular}}
  \fbox{\begin{tabular}{c}
            algebraic \\        
            function \\    
            fields 
        \end{tabular}} & \\
& & \\
& & \\
\fbox{\fbox{\begin{tabular}{c}
                matrices \\
                and \\
                vectors
            \end{tabular}}} & 
      \fbox{\fbox{arithmetic}} &
      \fbox{\fbox{polynomials}} \\
& & \\
& & \\
& \fbox{$\Z$}\ \ \fbox{$\Q$}\ \ \fbox{$\R$}\ \ \fbox{$\C$}\ \ 
 \fbox{$\Z/m\Z$}\ \ \fbox{$\Fq$} \fbox{$\Qp$} & \\
& & \\
\hline
\multicolumn{3}{|c|}{~} \\
\multicolumn{3}{|c|}{\fbox{\fbox{basic system}}} \\
\multicolumn{3}{|c|}{~} \\
\hline
\multicolumn{3}{|c|}{~} \\
\multicolumn{3}{|c|}{\fbox{\fbox{programming language C}}} \\
\multicolumn{3}{|c|}{~} \\
\hline
\multicolumn{3}{c}{~} \\
\hline
\multicolumn{3}{|c|}{~} \\
\multicolumn{3}{|c|}{\fbox{\fbox{Shell}}} \\
\multicolumn{3}{|c|}{~} \\
\hline
\end{tabular}
\end{center}

\newpage  \noindent 
SIMATH consists of 
\begin{itemize}
\item an {\em interface\/} between the operating system and SIMATH;
\item the {\em programming language} C;
\item the {\em basic\/} system which consists of 
modified input/output functions, and a {\em list\/} system with an 
automatic {\em garbage collector\/} and dynamic memory administration;
\item a {\em multiple precision arithmetic\/} package for computations over 
$\Z$, $\Q$, $\R$, $\C$, $\Z/m\Z$, $\Qp$, finite fields, and global 
fields, i.e. algebraic number fields and function fields;
\item a {\em polynomial\/} package for computations with polynomials in any number
of unknowns over any of the structures contained in the arithmetic package;
\item a {\em matrix--vector\/} package for matrix/vector computations over the structures
contained in the arithmetic package and over polynomial rings;
\item an {\em elliptic curves\/} package with elliptic-curve-specific functions over
the rational numbers, prime fields, finite fields of characteristic 2 and 
algebraic number fields;
\item the interactive SIMATH calculator {\em simcalc\/}.
\end{itemize}
The {\em number theory\/} package contains higher algorithms for algebraic number
theory such as 
\begin{itemize}
\item integral bases, extension of valuations, and the decomposition 
law for number fields and congruence function fields;
\item for quadratic congruence function fields: regulators, unit groups,
divisor and ideal class number, and generators and type of isomorphism of the
ideal class group and the zero class group;
\item conductor, minimal model, and an algorithm for finding the rank and a basis
of an elliptic curve over rational numbers, as well as Tate's algorithm over the
rational numbers and quadratic number fields;
\item combined Schoof-Shanks algorithm for counting points on elliptic curves
over prime fields and finite fields of characteristic 2;
\item an algorithm for constructing elliptic curves with a given number of points
over a given prime field;
\item LLL-algorithm.
\end{itemize}

The system libraries each consist of a package of functions which
-- except for some internal initialization and managing procedures -- 
can be integrated into any C pro\-gram by a simple function call.

\section{The installation of SIMATH}\index{SIMATH} 

The SIMATH system is installed by Makefiles and shell scripts. You will need 
about 50 MB disk space while compiling SIMATH. 
On all platforms the GNU readline library will 
be used in simcalc. This has many advantages such as emacs style command line 
editing and function name completion. Starting with Version 3.10.3, the 
readline library is part of the SIMATH distribution (see ./readline-2.0) and 
compiled automatically. 

  The latest version of SIMATH\index{SIMATH} and more information about installing the 
system may be obtained by anonymous ftp from simath.math.metro-u.ac.jp 
(133.86.76.12) in the directory /simath/latest.

% \end{document}
