% \documentstyle[11pt]{book1}
% \setlength{\voffset}{-2cm}
% \oddsidemargin0.5cm   \evensidemargin0.5cm
% \parindent0pt
% \textwidth 6.0in \textheight 21.8cm
% \input texdefs.tex
% \begin{document}
% \pagestyle{plain}
% \setcounter{chapter}{2} % one less than the intended chapter number

\chapter{Programming in SIMATH}

%%%%%%%%%%%%%%%%%%%%%%%%%%%%%%%%%%%%
%%%%%%%                      %%%%%%%
%%%%%%%   c h p 3 1 . t e x  %%%%%%%
%%%%%%%                      %%%%%%%
%%%%%%%%%%%%%%%%%%%%%%%%%%%%%%%%%%%%

\section{The extended programming language C}
The programs of the SIMATH user and the internal functions of the 
SIMATH system are, essentially, C programs. C was
extended by adding
\begin{itemize}
\item {\bf include files} for the preprocessor (\S 3.2),
\item {\bf data types} (\S 3.3),
\item {\bf format arguments} to {\em printf\/} and {\em fprintf\/} for
the new data types\\
(see on-line documentation of printf and fprintf, e.g. with @(f)printf),
\item {\bf functions}, collected in libraries (\S 3.6 and ch.~4),
\item {\bf global parameters} (\S 3.7).
x\end{itemize}
There are some changes from standard C programs in the declaration and instruction
parts of a function due to the new data types (\S 3.3).

%%%%%%%%%%%%%%%%%%%%%%%%%%%%%%%%%%%%
%%%%%%%                      %%%%%%%
%%%%%%%   c h p 3 2 . t e x  %%%%%%%
%%%%%%%                      %%%%%%%
%%%%%%%%%%%%%%%%%%%%%%%%%%%%%%%%%%%%
 
\section{Include files in SIMATH}
In order to develop SM programs, you must have the preprocessor instruction
\begin{center}
{\bf \# include $<$\_simath.h$>$}
\end{center}
before using SIMATH data types and SIMATH functions. {\bf \_simath.h} makes all of the include files of
the SIMATH system and the include files of the C libraries which are required by SIMATH
available.

The include files of the SIMATH system contain
\begin{itemize}
\item {\em define} instructions for the implementation of 
      {\em global constants\/},
\item {\em macros\/} for the different program packages,
\item {\em type definitions\/},
\item declaration of {\em global parameters\/}.
\end{itemize}
The following C libraries are automatically included into any SM program via
{\bf \_simath.h}:
\begin{tabular}{ll} 
$<$stdio.h$>$ &\\ 
$<$setjmp.h$>$ &\\ 
$<$ctype.h$>$ &\\
$<$math.h$>$ &\\
$<$sys/types.h$>$&\\
$<$sys/times.h$>$&\\
$<$sys/param.h$>$&\\
$<$sys/stat.h$>$&.\\
\end{tabular} 

%%%%%%%%%%%%%%%%%%%%%%%%%%%%%%%%%%%%
%%%%%%%                      %%%%%%%
%%%%%%%   c h p 3 3 . t e x  %%%%%%%
%%%%%%%                      %%%%%%%
%%%%%%%%%%%%%%%%%%%%%%%%%%%%%%%%%%%%

\newpage
\section{Data types in SIMATH}
\index{data types in SIMATH}
\subsection{C data types}
\index{C data types}
The data types of C remain valid in SIMATH.\\
Exception: Instead of {\bf int}, integers with bounded 
values have the type de\-sig\-nation {\bf single} ({\em single precision 
integer\/}).

Variables {\bf x} of type {\bf single} are bounded by
$$ |x| < 2^{30} .$$
Variables of type {\bf single} are subject to the same conditions, rules,
and ap\-pli\-ca\-tions as variables of type {\bf int} in C. For reasons of program
readability, the type modifiers {\bf short} and {\bf long} should be avoided.

\subsection{Additional data types in SIMATH}
\index{Additional data types in SIMATH}
Here is a short description of each new data type and in which part of SIMATH
it is defined.
\leer
\begin{tabular}{lll}
Type & Description & Package\\
\cline{1-3} \\
{\bf CELL}\index{CELL}\footnotemark[1] & cell: structure (see \S 4.1.3) & Base\\
{\bf PCELL}\index{PCELL}\footnotemark[1] & pointer to a cell & Base\\
{\bf atom}\index{atom} & integer $x$ with $|x|<2^{30}$ & Base\\    
{\bf obj}\index{obj} & object: atom or list & Base\\    
{\bf list}\index{list} & pointer to a list of objects & Base\\    
{\bf single}\index{list} & single precision integer ($x$ with $|x|<2^{30}$) & Arithmetic\\ 
{\bf int}\index{int} & integer of arbitrary size & Arithmetic\\    
{\bf rat}\index{rat} & rational number & Arithmetic\\    
{\bf floating}\index{floating} & real number & Arithmetic\\
{\bf complex}\index{complex} & complex number & Arithmetic\\
{\bf gfel}\index{gfel} & element of a Galois field & Arithmetic\\    
{\bf gf2el}\index{gf2el} & element of a Galois field of characteristic 2 & Arithmetic\\    
{\bf nfel}\index{nfel} & element of a number field & Arithmetic\\
{\bf pfel}\index{pfel} & $p$-adic number & Arithmetic\\    
{\bf rfunc}\index{rfunc} & rational function & Arithmetic\\    
{\bf afunc}\index{afunc} & algebraic function & Arithmetic\\
{\bf pol}\index{pol} & polynomial & Polynomial\\    
{\bf matrix}\index{matrix} & matrix & Matrix-vector\\    
{\bf vec}\index{vec} & vector & Matrix-vector\\   
\end{tabular}

\footnotetext[1]{used only by the system}   

Each new data type will be described in detail in chapter 4.

\newpage
\subsection{Definitions using SIMATH data types}
\index{SIMATH-type variables, - definition rules}
These are the rules per\-tai\-ning to variables of the type described in
\S 3.3.2 (except for {\bf atom}); we will refer to those variables as
``SIMATH-type'' variables. Ignoring these rules may lead to inconsistencies
in the SIMATH memory administration. The rules apply analogously to any
new type definitions introduced by {\bf typedef}.

\begin{tabular}{rp{11.5cm}} 
1. & The C instructions {\bf longjmp()}, {\bf setjmp()}, {\bf break}, 
	{\bf continue}, and {\bf goto} should not be used within the scope of 
	SIMATH-type variables.\\
2. & SIMATH-type variables cannot be declared with the {\bf static} storage
      class specifier. Exception: see the on-line documentation for {\bf globinit()}.\\
3. &  SIMATH-type variables cannot be declared as an element of a {\bf union} definition.\\
4. & SIMATH-type variables (of the same type) cannot be initialized simultaneously
       (as is often done in C programs).\\
\end{tabular}

\subsection{The {\bf bind()} and {\bf init()} instructions}
\index{bind()}\index{init()}
The {\bf bind()} and {\bf init()} in\-struc\-tions are used between the declaration 
and instruction parts of a main function or block to create
a reference in the SIMATH stack for each SIMATH-type variable; variables which have
no reference in the stack will be returned by the garbage collector to the list of
available cells (thus their contents will be destroyed). These are {\em independent\/}
instructions, i.e. they must be terminated by a semicolon and a line feed is not allowed
within the instruction. {\bf init()} differs from {\bf bind()} in that it also initializes
each parameter to its corresponding nil value.

For any SIMATH function having SIMATH-type variables {\bf $x_1, ..., x_m$}
as pa\-ra\-me\-ters and SIMATH-type variables {\bf $y_1, ..., y_n$} in the declaration
part, we must use the instructions
\begin{center}
{\bf bind($x_1,...,x_m$);} \\
{\bf init($y_1,...,y_n$);}
\end{center}
before the instruction part of the function. \\
{\bf Note} that we do {\bf not} use {\bf init($x_1,...,x_m$)}
since arguments used to call a function usually already have a value assigned to them!

For a main function, all SIMATH-type global variables must be ``initialized'' by
{\bf init()} or {\bf globinit()} before the instruction part.

\newpage

For structures, e.g.
   {\tt
    \begin{tabbing}
    1234567890\=123456\=12345\=\kill
    \>    typedef struct \{ \> \\
    \> \>    char NAME; \\
    \> \>    list L;    \\
    \>   \} nlist; \> \\
    \> \\   
    \>    nlist S;   \> \\
    \>    nlist *p = \&S; \> \\
    \> \\   
    \> \\   
    \>    struct T \{ \> \\
    \> \>   list A; \\
    \> \>   nlist NL; \\
    \>   \} 
    \end{tabbing}  
   }
{\bf init()} and {\bf bind()} must be used for each component which is of
SIMATH-type; e.g.
   {\tt
    \begin{tabbing}
    1234567890\=123456\=12345\=\kill
    \>  init(T.A, p-$>$L, T.NL.L);\> 
    \end{tabbing}  
   }

For SIMATH-type arrays, e.g.
   {\tt 
    \begin{tabbing}
    1234567890\=1234\=\kill
    \>  pol PV[10];\\
    \>  int IV[3][100];
    \end{tabbing}
   }
{\bf init()} and {\bf bind()} must be used for each component; e.g.
   {\tt 
    \begin{tabbing}
    1234567890\=1234\=\kill
    \>  init(PV[0..9], IV[0..2][0..99]);  
    \end{tabbing}
   }

The values in the square brackets must be constants or variables; e.g.
   {\tt
    \begin{tabbing}
    1234567890\=123\=\kill
    \>  up(n)\\
    \>  single n;\\
    \>  \{ \\
    \>  \ \ \ \ \  int VEC[1000];\\
    \>  \ \ \ \ \  init(VEC[0..n]);\\
    \>  \ \ \ \ \ \ \ \ \ \ \vdots\\
    \>  \}
    \end{tabbing}
   }

\newpage

In order not to overload the reference list of the SIMATH memory ad\-mi\-nistra\-tion,
in some specific cases, {\bf init()} and {\bf bind()} can be called with
single components of structures or partial areas of arrays:
\begin{itemize}
\item if you are sure that {\bf only} those components or array areas will
      be used throughout the current computation,
\item if you know that the other components or array areas have already been
      ``bound'' or ``initialized'' in higher calling functions,
\item for arrays of type {\bf int}, if only {\bf single} values will occur
      in that array area throughout the current computation,
\end{itemize}
i.e. statements such as
   {\tt
    \begin{tabbing}
    1234567890\=123\=\kill
    \> init(PV[2], IV[2][20]);
    \end{tabbing}
   }
and
   {\tt
    \begin{tabbing}
    1234567890\=123\=\kill
    \> init(IV[0..2][0..19], IV[0][20..99]);
    \end{tabbing}
   }
are possible.

When using {\bf bind()} and {\bf init()} on single components of an array, 
pointer notation is allowed; e.g. one can write
   {\tt 
    \begin{tabbing}
    1234567890\=123\=\kill
    \>  init(*(PV+2), (*(IV+2))[20]);    
    \end{tabbing}
   }
for
   {\tt 
    \begin{tabbing}
    1234567890\=123\=\kill
    \>  init(PV[2], IV[2][20]);    
    \end{tabbing}
   }


Finally, for variables defined as a mixture of arrays and structure components,
e.g.
   {\tt
    \begin{tabbing}
    1234567890\=123456\=12345\=\kill
    \>    typedef struct \{ \> \\
    \> \>    list paar; \\
    \> \>    list s;    \\
    \>   \} feld[10]; \> \\
    \> \\   
    \>    feld str1;   \> \\
    \> \\   
    \> \\   
    \>    struct name \{ \> \\
    \> \>   feld strfeld; \\
    \> \>   char lname[20]; \\
    \>   \} 
    \end{tabbing}  
   }
{\bf bind()} and {\bf init()} must be called as follows:
   {\tt 
    \begin{tabbing}
    1234567890\=123\=\kill
    \>  bind(str1[1..n].paar);\\
    \>  init(name.strfeld[0].s);
    \end{tabbing}
   }
``Binding'' and ``initializing'' whole array areas can be done only
for the first component of a structure, i.e.
   {\tt 
    \begin{tabbing}
    1234567890\=1234\=\kill
    \>   init(name.strfeld[1..n].s);
    \end{tabbing}
   }
is not allowed.

Note: {\bf bind()} and {\bf init()} are interpreted by the SIMATH-preprocessor.
To avoid the SIMATH-preprocessor you can use a combination of 
{\bf Sbind()-Sfree()} or {\bf Sinit()-Sfree()}. See the documentations to
these functions.

\subsection{Computations with SIMATH data types}
All operations which refer to SI\-MATH data types are im\-ple\-men\-ted using 
SIMATH functions, i.e.\/ C operators must be replaced by the corresponding
SIMATH functions. For example, the C instruction
\begin{center} 
{\bf  c = a $*$ b;}
\end{center}
for integers {\bf a}, {\bf b}, and {\bf c} must be replaced by
\begin{center} 
{\bf  c = iprod(a, b);}
\end{center}
in a SIMATH program.

Note: In order to avoid strange behavior in your SIMATH programs, always
be wary of nested function calls which return SIMATH-type results; see \S 3.5.


%%%%%%%%%%%%%%%%%%%%%%%%%%%%%%%%%%%%
%%%%%%%                      %%%%%%%
%%%%%%%   c h p 3 4 . t e x  %%%%%%%
%%%%%%%                      %%%%%%%
%%%%%%%%%%%%%%%%%%%%%%%%%%%%%%%%%%%%

\newpage
\section{Example}
We now give an example of a SIMATH program; we will modify the functions {\bf gcd()},
{\bf lcm()} and {\bf mainp()} from \S 2.2. A detailed description of the modi\-fi\-ca\-tions
will be given in chapter 4.

Now change the file {\bf gcd.S} as follows:
\begin{verbatim}
#include <_simath.h>

int             gcd(a, b)
    int             a, b;
{
    /* declaration part */
    int             c;

    /* binding and initialization */
    bind(a, b);
    init(c);

    /* instruction part */

    if (icomp(a, b) < 0)
      { c = a; a = b; b = c; }

    while (b != 0)
      { c = irem(a, b); a = b; b = c; }

    return (a);
}
\end{verbatim}
Some remarks about the modi\-fi\-ca\-tions: as mentioned in \S 3.2, we included
the SIMATH system file {\bf $<$\_simath.h$>$}. We added the {\bf bind()} and
{\bf init()} functions (as seen in \S 3.3.4) and the {\bf icomp()}  and  {\bf irem()}
functions from the arithmetic package (described in \S 4.2.4 ). Note that equality
comparison with zero can be carried out with the usual C operators, i.e. == and !=,
independently of the data type (see \S 4.2).

\newpage

Now let us modify the file {\bf lcm.S}.
\begin{verbatim} 
#include <_simath.h>

int             lcm(a, b)
    int             a, b;
{
    /* declaration part */
    int             c;

    /* binding and initialization */
    bind(a, b);
    init(c);

    /* result zero */
    if (a == 0 || b == 0)
        c = 0;
    else
    {
        /* general case */
        a = iabs(a);
        b = iabs(b);

        /* c = (a / gcd(a,b)) * b */
        c = iprod(iquot(a, gcd(a, b)), b);
    }
    return (c);
}
\end{verbatim}  
Some more remarks: here we added the SIMATH functions {\bf iabs()}, {\bf iprod()}, and
{\bf iquot()} all from the arithmetic package. Note that this time we do not need
to return a replacement value for non-representable results.

\newpage
 
Finally change {\bf mainp.S}.
\begin{verbatim}
#include <_simath.h>

main()
{
    /* declaration part */
    int             a, b;

    /* binding and initialization */
    init(a, b);

    /* read the numbers */
    printf("\nEnter the numbers: \na = ");
    while ((a = geti()) == ERROR)
        printf("Reading error \na = ");
    printf("b = ");
    while ((b = geti()) == ERROR)
        printf("Reading error \nb = ");

    /* lcm */
    printf("lcm(%i, %i) = %i \n", a, b, lcm(a, b));
}
\end{verbatim} 
Last remarks: since we are using the SIMATH function {\bf geti()} to read in
the numbers, we do not need to change the symbol order to numbers as in
\S 2.2. Note the use of the argument \%i in the {\bf printf()} command instead
of \%d.

Do not forget to compile and archive the modified {\bf gcd()} and {\bf lcm()}
functions, and to compile and link the modified {\bf mainp()} function as 
described in \S 2.2.

Now carry out the computation in the example session of \S 2.2 which 
printed an error message.
\leer
{\tt
$<$ R \care \\
RUN "mainp"\\[1.0eM]
Enter the numbers:\\
a = {\bf 44444 \care}\\
b = {\bf 10000000 \care}\\
lcm(44444, 10000000) = 111110000000
}
\begin{verbatim}
***  # GCs: 1     GC time: 0.00 s    # collected cells: 0      ***    
***  # blocks: 1  block size: 16383  # free cells:      16369  ***    
***  total CPU time: 0.01 s                                    ***      
\end{verbatim}
$<$
\leer
As you can see, computing with large numbers is no problem in SIMATH.
At the end of the program, memory administration information and
computation time are given. In the above example, most of the time is used
by the initialization of the SIMATH memory administration; only a small fraction
of the time is used for the actual computation.

In the directory {\bf /usr/local/simath/examples/basics/} you can find further
examples of simple SIMATH-programs.


%%%%%%%%%%%%%%%%%%%%%%%%%%%%%%%%%%%%
%%%%%%%                      %%%%%%%
%%%%%%%   c h p 3 5 . t e x  %%%%%%%
%%%%%%%                      %%%%%%%
%%%%%%%%%%%%%%%%%%%%%%%%%%%%%%%%%%%%

\section{Nested function calls}
In general, nested function calls are permitted in SIMATH. For example,
for integers {\bf M}, {\bf N}, and {\bf P}, the assignment statement
\begin{center}
{\bf P = (M+7) $\cdot$ N}
\end{center}
is replaced by
\begin{center}
             {\bf P = iprod(isum(M,7), N)}.
\end{center}

Nested function calls will not cause any problems if at most one of 
the parameters is a function or a macro.

In the current SIMATH version, the nesting of functions which return SIMATH-type
results (except for {\bf atom}) might cause inconsistencies in the dynamic memory
administration. 
%This problem occurs only if the automatic garbage collector
%{\bf gc()} is started while the return value of a function  or a macro is evaluated.
%An enlarged list store and a explicitly started {\bf gc()} may prevent this.
%This limitation shall be removed in a later SIMATH version.

In general, macros should never be called with a function or a macro as a parameter. 
If you want to nest macros, you should check how the macros are expanded by the C
compiler; use the {\bf (p)re} command (see \S 2.4.2) to see how the C preprocessor
expands them.

A tool which can help you to unnest nested function calls by automatically
inserting temporary variables in a SIMATH source file is distributed with 
the SIMATH package. See the directory packages/unnest/ for details.  


%%%%%%%%%%%%%%%%%%%%%%%%%%%%%%%%%%%%
%%%%%%%                      %%%%%%%
%%%%%%%   c h p 3 6 . t e x  %%%%%%%
%%%%%%%                      %%%%%%%
%%%%%%%%%%%%%%%%%%%%%%%%%%%%%%%%%%%%

\newpage
\section{Nomenclature of SIMATH functions}\index{nomenclature}
In every SIMATH function, you will find a title containing a few keywords which briefly 
describe what the function does, e.g.

\begin{tabbing}
     \ \ \ \ \ \ \ \ \ \   \=          {\bf integer product,}  \=\\
     \>{\bf polynomial over modular singles resultant.} 
\end{tabbing} 

Suitable abbreviations of these titles have been chosen as function 
names, e.g. for the examples above

\begin{tabbing} 
\ \ \ \ \ \ \ \ \ \  \= {\bf iprod(),} \= \\    
\>{\bf pmsres().}
\end{tabbing}

Because of our strict nomenclature rules, you will know the most important function
names after using the system for only a short time.

As examples of those rules, here is a short list of function names and their application area.
\leer
\begin{tabular}{p{1.0in}p{1.3in}p{3.0in}}
 Package     & Function name & Function application \\ \hline
             &               &  \\
 Base        & l\ldots       & list \\
             & o\ldots       & object \\
             & s\ldots       & set \\
             & us\ldots      & unordered set \\
             &               &  \\ 
 Arithmetic  & cset\ldots    & characteristic set \\
             & s\ldots       & single precision \\
             & i\ldots       & integer \\
             & s\ldots      & single-precision integer \\
             & ms\ldots      & modular single \\
             & mi\ldots      & modular integer \\
             & r\ldots       & rational \\
             & fl\ldots      & floating point number\\
             & c\ldots       & complex number \\
             & nf\ldots      & number field \\
             & qnf\ldots     & quadratic number field \\
             & gfs\ldots     & Galois field of single characteristic \\
             & gf2\ldots     & Galois field of characteristic 2 \\
             & pf\ldots      & $p$-adic field \\ 
             & qff\ldots     & quadratic function field \\     
             & rfmsp\ldots   & rational function over modular single primes \\
             & rfr\ldots     & rational function over the rationals \\
             & afmsp\ldots   & algebraic function over modular single primes
\end{tabular}
\newpage
\begin{tabular}{p{1.0in}p{1.3in}p{3.0in}}
 Package           & Function name & Function application \\ \hline
                   &               & \\
 Elliptic curves   & ec\ldots      & elliptic curve (general)\\
                   & eci\ldots     & \ldots with integer coefficients\\
                   & ecr\ldots     & \ldots over the rationals\\
                   & ecnf\ldots    & \ldots over number field\\
                   & ecqnf\ldots   & \ldots over quadratic number field\\
                   & ecmp\ldots    & \ldots over modular primes\\
                   & ecgf2\ldots   & \ldots over Galois field of characteristic 2\\
                   &               & \\
                   & ec$\ast$ac\ldots   & \ldots , actual model\\
                   & ec$\ast$snf\ldots  & \ldots , short normal form\\
                   & ec$\ast$min\ldots  & \ldots , minimal model\\
                   &               & \\
 Polynomial        & p\ldots       & polynomial (general)\\
                   & pi\ldots      & \ldots over integers\\
                   & pms\ldots     & \ldots over modular single\\
                   & pmi\ldots     & \ldots over modular integer\\ 
                   & pr\ldots      & \ldots over the rationals\\
                   & pfl\ldots     & \ldots over floating point numbers\\
                   & pc\ldots      & \ldots over complex numbers\\
                   & pnf\ldots     & \ldots over number field\\
                   & pgfs\ldots    & \ldots over Galois field of single characteristic\\
                   & pgf2\ldots    & \ldots over Galois field of characteristic 2\\
                   & ppf\ldots     & \ldots over p-adic field\\
                   & prfmsp\ldots  & \ldots over rational functions over modular single primes \\
                   &               & \\
                   & dip\ldots     & distributive polynomial\\
                   & dp\ldots      & dense polynomial\\

                   & up\ldots      & univariate polynomial\\
                   & udp\ldots     & univariate dense polynomial\\
                   &               & \\
 Matrix            & ma\ldots      & matrix (general) \\ 
                   & mai\ldots     & \ldots with integer entries\\
                   & mams\ldots    & \ldots with modular single entries\\
                   & mami\ldots    & \ldots with modular integer entries\\
                   & mam2\ldots    & \ldots with entries of $\Z/2\Z$ in special bit re\-pre\-sen\-ta\-tion\\
                   & mar\ldots     & \ldots with rational entries\\
                   & manf\ldots    & \ldots with number field entries\\
                   & magfs\ldots   & \ldots with Galois field of single characteristic entries\\
                   & magf2\ldots   & \ldots with Galois field of characteristic 2 entries
\end{tabular}
\newpage
\begin{tabular}{p{1.0in}p{1.3in}p{3.0in}}
 Package     & Function name & Function application \\ \hline
                   &               & \\
                   & maup\ldots    & matrix of univariate polynomials\\
                   & map\,/\,mp\ldots & matrix of polynomials (general) \\ 

                   & mapi\ldots    & \ldots over integers\\
                   & mapms\ldots   & \ldots over modular single\\
                   & mapmi\ldots   & \ldots over modular integer\\
                   & mapr\ldots    & \ldots over the rationals\\
                   & mapnf\ldots   & \ldots over number field\\
                   & mapgfs\,/\,mpgfs\ldots & \ldots over Galois field of single characteristic\\
                   & mapgf2\,/\,mpgf2\ldots & \ldots over Galois field of characteristic 2\\
                   &               & \\
                   & marfr\ldots   & \ldots of rational functions over the rationals\\
                   & marfmsp\ldots & \ldots of rational functions over modular\newline single primes\\
                   &               & \\
 Vector            & vec\ldots     & vector (general)\\
                   & veci\ldots    & \ldots with integer entries\\
                   & vecms\ldots   & \ldots with modular integer entries\\
                   & vecmi\ldots   & \ldots with modular single entries\\
                   & vecr\ldots    & \ldots with rational entries\\
                   & vecnf\ldots   & \ldots with number field entries\\
                   & vecgfs\ldots  & \ldots with Galois field of single characteristic entries\\
                   & vecgf2\ldots  & \ldots with Galois field of characteristic 2 entries \\
                   &               & \\
                   & vecup\ldots   & vector of univariate polynomials\\
                   & vecp\,/\,vp\ldots & vector of polynomials (general) \\ 

                   & vecpi\ldots   & \ldots over integers\\
                   & vecpms\,/\,vpms\ldots & \ldots over modular single\\
                   & vecpmi\,/\,vpmi\ldots & \ldots over modular integer\\
                   & vecpr\ldots   & \ldots over the rationals\\
                   & vecpnf\,/,vpnf\ldots & \ldots over number field\\
                   & vecpgfs\,/\,vpgfs\ldots & \ldots over Galois field of single characteristic\\
                   & vecpgf2\,/\,vpgf2\ldots & \ldots over Galois field of characteristic 2\\
                   &               & \\
                   & vecrfr\ldots  & \ldots of rational functions over the rationals\\
                   & vecrfmsp\ldots & \ldots of rational functions over modular single primes\\
\end{tabular}
\leer
\leer
Input and output functions are adapted to the nomenclature in C.

\begin{tabular}{p{1.3in}p{1.0in}p{3.0in}}
      $_{~}$\                        & get\ldots & \\
                                     & put\ldots & \\
                                     & fget\ldots & \\
                                     & fput\ldots \ \ \ & 
\end{tabular}

\newpage

Functions which serve to test certain conditions of the data 
(``is single precision?'', ``is integer?'') have the form

\begin{tabular}{p{1.3in}p{1.0in}p{3.0in}}
$_{~}$\                     & is.... & 
\end{tabular}

%%%%%%%%%%%%%%%%%%%%%%%%%%%%%%%%%%%%
%%%%%%%                      %%%%%%%
%%%%%%%   c h p 3 7 . t e x  %%%%%%%
%%%%%%%                      %%%%%%%
%%%%%%%%%%%%%%%%%%%%%%%%%%%%%%%%%%%%
\def\flu{\begin{flushleft}}
\def\ulf{\end{flushleft}}
\def\xxnewline{ }
\def\xxlinebreak{ }   
\def\rg{\raggedright}      

\section{Global parameters in SIMATH}\index{global parameters}
Here is a list of all the parameters used in SIMATH; parameters required for
programming in SIMATH and not mentioned here will be considered in detail in chapter 4.
Some parameters have a default value (d), others an initial value (i) which is
changed automatically by the system during program execution.

The abbreviations B, A, E, P, and M (in the second column of the table) indicate
in which package -- base, arithmetic, elliptic curves, polynomial, or matrix/vector -- 
the parameters find their application.

The third column ({\em owner\/}) indicates which parameters can be changed either
auto\-ma\-tically by the system or by the user; constants are identified as such and cannot
be changed.\\
{\bf NOTE}: The default value of parameters ``owned'' by user(E) (E for Expert)
should in NO CASE be changed by programmers with little or no programming experience.\\
The last column gives a short description of each parameter.

The system variable {\bf BL\_START[0]} points to the first of the memory blocks cur\-rent\-ly
allocated for the program execution; we shall call it the current program ``workspace''.
Within the workspace, {\bf AVAIL} points to the list of free memory cells.
\leer
\leer
{\small{
\begin{tabular}{p{1.65in}p{0.2in}|p{0.15in}|p{0.54in}|p{2.60in}} 
{Parameter, value (d)/(i)} &     &      & owner         & description \\ \hline  
                           &     &      &               & \\
{\bf AVAIL} = \_0    & (i) & B  & system        & list of free memory cells;\\
                           &     &      &               & \\ 
{\bf BASIS} $=2^{30}$      & (d) & B    & constant      & base for list representation of large \newline numbers;\\
                           &     &      &               & \\ 
{\bf BLOG10} $=9$          & (d) & A    & constant      & internal use: $\log_{10}$ {\bf BASIS}\\
                           &     &              &               & \\ 
{\bf BLOG2} $=30$          & (d) & A    & constant      & internal use: $\log_2$ {\bf BASIS}\\
                           &     &              &               & \\ 
{\bf BL\_NR} $=0$          & (i) & B    & system        & number of blocks currently allocated to the workspace;\\
                           &     &      &               & \\ 
{\bf BL\_NR\_MAX}          & (d) & B    & user(E)       & maximum number of blocks which can be allocated to the
                                                          workspace (the default value depends on your machine);
can be changed by gcreinit();\\
                           &     &              &               & \\ 
{\bf BL\_SIZE} $=2^{14}-1$ & (d) & B    & user(E)       & number of cells in a memory block; can be \\
                           &     &      &               & changed by gcreinit() (see on-line \\
                           &     &      &               & documentation);
\end{tabular}
\newpage
\begin{tabular}{p{1.65in}p{0.2in}|p{0.15in}|p{0.54in}|p{2.60in}} 
{Parameter, default value} &     &      & owner         & description \\ \hline  
                           &     &      &               & \\ 

{\bf BL\_START[\ ]}        &     & B    & system        & vector containing the starting address of each block allocated
                                                          to the workspace;\\
                           &     &       &              & \\ 
{\bf BSMALL} $=2^{15}$     & (d) & A    & constant      & internal use: multiplication of large in\-te\-gers; \\
                           &     &      &               & \\  
{\bf DECBAS} $=10^9$       & (d) & A    & constant      & decimal base for input and output of large numbers;\\
                           &     &      &               & \\
{\bf DIFF[481]}            & (d) & A    & constant      & differences between the units \newline  $a_i \pmod{2310}$, for $1009 \leq
                                                          a_i \leq 3319$;\\
                           &     &      &               & \\
{\bf DUM} $=0$             & (i) & B    & system        & internal dummy 
						variable: used as a par\-ameter 
						in some macros;\\
                           &     &      &               & \\
{\bf ERROR} $=-2^{30}$     &     & B    & constant      & return value of SIMATH functions in case of error;\\
                           &     &      &               & \\
{\bf FL\_EPS} $=5$         & (d) & A    & user          & maximal list length of the mantissa of a floating point number;\\
                           &     &      &               & \\
{\bf FL\_JMP}              & (i) & A    & system        & internal variable for handling overflow of floating point numbers;\\  
                           &     &      &               & \\
{\bf FL\_LN2} $=0$    & (i) & A         & system        & internal variable for floating point arith\-me\-tic;\\
                           &     &      &               & \\
{\bf FL\_STZ} = {\bf ST\_INDEX}
                           & (i) & A    & system        & internal variable for handling overflow of floating point numbers;\\
                           &     &      &               & \\
{\bf GC\_CC} $=0$          & (i) & B    & system        & counter: number of cells collected by the \newline
                                                          garbage collector;\\
                           &     &      &               & \\ 
{\bf GC\_COUNT} $=0$       & (i) & B    & system        & counter: number of garbage collections;\\
                           &     &      &               & \\
{\bf GC\_MESS} $=0$        & (d) & B    & user          & messages from the garbage collector; \newline suppress = 0 / print = 1;\\
                           &     &      &               & \\
{\bf GC\_QUOTE} $=10$      & (d) & B    & user(E)       & If the percentage of 
                free cells with respect to the size of the current workspace is 
                less than 1/{\bf GC\_QUOTE}, the workspace is enlarged; 
                if {\bf BL\_NR} $=$ {\bf BL\_NR\_MAX}, the garbage collector 
                will terminate the program ex\-ecu\-tion; 
\end{tabular}
\newpage
\begin{tabular}{p{1.65in}p{0.2in}|p{0.15in}|p{0.54in}|p{2.60in}} 
{Parameter, default value} &     &      & owner         & description \\ \hline  
                           &     &      &               & \\
{\bf GC\_TEST} $=0$        & (d) & B    & user          & the garbage collector will perform some additional tests on the program's
                                                          lists; \newline disable = 0 / enable = 1. This can be used as a debugging tool
                                                          (see on-line documentation on {\bf islist()});\\
                           &     &      &               & \\
{\bf GC\_TIME} $=0$        & (i) & B    & system        & total CPU time required for all garbage \newline
                                                          collections;\\
                           &     &      &               & \\ 
{\bf \_H\_BOUND} $=0.0$    & (i) & E    & user          & if \_H\_BOUND$>0.0$, 
					it is used for the search for points in 
					the algorithm of Manin as the upper 
					limit for the Weil height of points on 
					an elliptic curve over Q; 
                        		if \_H\_BOUND$=0.0$, the algorithm 
					searches for points with unlimited Weil 
					height; for most computations 
                        		\_H\_BOUND$=11.0$ is sufficient; \\
                           &     &      &               & \\ 
{\bf ITERMAX} $=500$       & (d) & P    & user          & maximal number of iteration steps: used in computation of the real and complex roots of a polynomial
(udprf.S);\\
                           &     &      &               & \\
{\bf LIST\_GSP[51]}        & (d) & P    & constant      & list of the 50 largest single precision primes: used in polynomial
                                                          factorization;\\  
                           &     &      &               & \\
{\bf LIST\_SP[169]}        & (d) & A    & constant      & list of all primes smaller than 1000: used in integer factorization;\\
                           &     &      &               & \\ 
{\bf LN\_SIZE} $=80$       & (d) & B    & user          & line length for output;\\
                           &     &      &               & \\ 
{\bf MADUMMY} $=0$         & (i) & M    & system        & internal dummy variable: used as a par\-ameter in some macros for matrix com\-pu\-ta\-tions; \\
                           &     &      &               & \\ 
{\bf MANUMMY} $=0$         & (i) & M    & system        & internal dummy variable: used as a par\-ameter in some macros for matrix com\-pu\-ta\-tions; \\
                           &     &      &               & \\ 
{\bf MARGIN} $=0$          &(d)  & B    & user          & left margin: column number where output should start;\\
                           &     &      &               & \\ 
{\bf NUM} $=0$             &(i)  & B    & system        & internal dummy variable: used as a par\-ameter in some macros;\\
                           &     &      &               & \\ 
{\bf POLDUMMY} $=0$        & (i) & P    & system        & internal dummy variable: used as a par\-ameter in some macros for polynomial com\-pu\-ta\-tions; 
\end{tabular}
\newpage
\begin{tabular}{p{1.65in}p{0.2in}|p{0.15in}|p{0.54in}|p{2.60in}} 
{Parameter, default value} &     &      & owner         & description \\ \hline  
                           &     &      &               & \\ 
{\bf RESTORE}              &     & B    & system        & internal dummy variable for the preprocessor: used with the return value
                                                          of functions;\\
                           &     &      &               & \\ 
{\bf SEMIRAND}             &     & A    & user          & controls the behaviour of irand(); \\
                           &     &      &               & \\
{\bf SP\_MAX} = \newline{\mbox{\hspace{5pt}{\bf BL\_SIZE}}}\newline{\mbox{\hspace{40pt}$*$ {\bf BL\_NR\_MAX}}}
                           & (d) & B    & user(E)          & maximum number of cells which can be allocated to the workspace of a
                                                          running program; can be changed by gcreinit(); \\
                           &     &      &               & \\ 
{\bf STACK}                &     & B    & system        & pointer to the SIMATH stack of references;\\
                           &     &      &               & \\ 
{\bf ST\_INDEX } $=0$      & (i) & B    & system        & stack pointer to the last occupied position in the SIMATH stack;\\
                           &     &      &               & \\ 
{\bf ST\_SIZE} $=500$      & (d) & B    & user          & size of the SIMATH stack; can be changed via the {\bf setstack()} instruction;\\
                           &     &      &               & \\
{\bf \_0}                  &     & B    & constant      & empty list;
\end{tabular} 
}}

% \end{document}
