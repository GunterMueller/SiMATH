\documentclass{article}

\addtolength{\textwidth}{18pt}

\begin{document}

\section{SMif --- The SIMATH interface functions}

SMif is the SIMATH interface program. SMif uses a simple script
language to extract data from arbitrary text files or from the output
of arbitrary programs. In some sense it is similar to the standard
Unix utilities awk or sed although some of the capabilities of these
languages have not been implemented and the design of SMif focuses on
the extraction of numerical data, and not on general text pieces.

SMif uses the \emph{tcl} library and numerical operations and internal
representations depend on \emph{tcl}.

\subsection{SMif functions}

The following funtions resp.\ macros can be used to obtain data from
the underlying SMif interpreter and to administrate the results of the
interpreter. All these functions can be used in any C program, in
particular in SIMATH programs. Since all of the \emph{tcl}
functionality resides in the interpreter, programs using SMif do not
have to be linked against the \emph{tcl} library.

\begin{verbatim}
  void SMifRes_Init (SMifRes* smifres)

  void SMifRes_FromFile (SMifRes smifres, char* textfname, 
                         char* scriptname)

  void SMifRes_FromProg (SMifRes smifres, char* progname,
                         char* paralist, char* scriptname)

  single SMifRes_Count (SMifRes smifres)

  single SMifRes_Error (SMifRes smifres)

  char* SMifRes_ErrMsg (SMifRes smifres)

  single SMifRes_Type (SMifRes smifres, single nr)

  char* SMifRes_Get (SMifRes smifres, single nr)

  void SMifRes_SetCsh (SMifRes smifres, char* cshname)

  void SMifRes_Free (SMifRes *smifres)
\end{verbatim}

Here, \texttt{SMifRes} is a data type used to store the results
returned by the SMif interpreter.

Every variable of type \texttt{SMifRes} must be initialized by a call
to \texttt{SMifRes\_Init()} before it can be used in any of the SMif
functions.  \texttt{SMifRes\_Free()} frees the storage area allocated
by \texttt{SMifRes\_Init()} for an object of type \texttt{SMifRes}.

\texttt{SMifRes\_FromFile()} requires as parameters the name of a text
file \texttt{textfname} and the name of a script file
\texttt{scriptname}.  The interpreter SMif operates on the text file
\emph{textfname} according to the script file \texttt{scriptname}. The
results are stored in the variable \texttt{smifres}.

\texttt{SMifRes\_FromProg()} expects as parameters the name of an
external program \texttt{progname}, the argument list
\texttt{paralist} for this program, and the name \texttt{scriptname}
of an SMif script. The external program \texttt{progname} is executed
for the arguments in \texttt{paralist} and the SMif interpreter
operates on its output according to the scriptfile
\texttt{scriptname}. The results are stored in the variable
\texttt{smifres}.

SMif creates some temporary \texttt{csh} scripts. By default, these
files are processed by \texttt{/bin/csh}. This can be changed by a
call to the \texttt{SMifRes\_SetCsh()} function.

The rest of the functions resp.\ marcos are used to read the results
of the SMif interpreter in a convenient manner.
\texttt{SMifRes\_Count()} returns the number of entries in a
\texttt{SMifRes} object. A valid entry can be of one of the types
\texttt{SM\_single}, \texttt{SM\_int}, \texttt{SM\_list}, or
\texttt{SM\_pol}.  \texttt{SMifRes\_Type()} returns the type of entry
\texttt{nr}. If this entry does not exist, \texttt{SM\_unknown} is
returned.  \texttt{SMifRes\_Get()} accesses the value of the entry
\texttt{nr}, i.\ e.\ a single precision number, an integer of
arbitrary size, a SIMATH list, or a polynomial in the appropriate
external representation of these types.  (I.\ e.\ the
\texttt{SMifRes\_Get()} returns a string which can be easily converted
to the appropriate SIMATH data type by certain SIMATH functions.)

If SMif encounters an error, the macro \texttt{SMifRes\_Error()}
returns \texttt{SMifRes\_error}. (Otherwise \texttt{SMifRes\_ok} is
returned.)  In case of an error \texttt{SMifRes\_ErrMsg()} returns a
detailed description of the error.

The functions, macros, and constants described above are declared
resp.\ defined in the header file \texttt{\_SMfiRes.h}.

\vspace*{12pt} \textbf{Example.} Let \texttt{anyprogram} be an
external program (i.\ e.\ a program which could as well be executed by
a shell) which returns the output

\begin{verbatim}
  *** anyprogram
  %1 = (x + y^3)(x^3 + y)
\end{verbatim}

for the arguments (in the string) \texttt{arg1 arg2}. Suppose that the
script file \texttt{script} looks as follows:

\begin{verbatim}
  SKIP UNTIL SEEN {[CHAR] =}
  GET POL
\end{verbatim}

We examine the following code fragment:

\begin{verbatim}
#include <_simath.h>
#include <_SMifRes.h>

int main() 
{
  SMifRes smifres;
  char *tmpstr = NULL;

  .....

  /* bind(), init() for SIMATH data types */

  .....

  /* Initialize a SMifRes variable */
  SMifRes_Init(&smifres);  
  
  SMifRes_FromProg(smifres, "anyprogram", "arg1 arg2", "script");
  
  if(SMifRes_Error(smifres) != SMif_ok) {
    /* Error processing */
    printf("SMifRes_Error: %s\n", SMifRes_ErrMsg(smifres));
    exit(1);
  }

  if(SMifRes_Count(smifres) != 1 || 
     SMifRes_Type(smifres, 1) != SM_pol) {
    /* Something went wrong */
    printf(.....);
    exit(1);
  }

  tmpstr = SMifRes_Get(smifres, 1);
  printf("example: %s\n", tmpstr);

  .....
  
  SMifRes_Free (&smifres);
 
  .....
}
\end{verbatim}

A part of the output of the program:

\begin{verbatim}
  example: 1*x^4*y^0 + 1*x^1*y^1 + 1*x^3*y^3 + 1*x^0*y^4
\end{verbatim}

\subsection{The SMif script language}

\subsubsection{Characters and words.}

SMif commands operate on a given text file. A pointer always points at
a well defined position in the text file. The text file is splitted in
appropriate tokens by default symbols or user defined symbols. The
following notation is used to describe characters and words.

\begin{verbatim}
  {[CHAR] ch1 ch2 ... [WORD] word1 word2 ...}
\end{verbatim}

Strings of this form are called \emph{controldefinestrings}. The
following command tells the interpreter to ignore all of the given
characters and strings within a text file to be processed by SMif.

\texttt{IGNORE} \emph{controldefinestring}

\subsubsection{\texttt{GET} commands}

The four \texttt{GET} comands \texttt{GET SINGLE}, \texttt{GET INT},
\texttt{GET LIST}, and \texttt{GET POL} generate the output of the
interpreter.  The \texttt{GET SINGLE}, \texttt{GET INT}, and
\texttt{GET LIST} command search for integer expressions beginning
with the current pointer position. A \texttt{GET POL} command collects
all the items which are not to be ignored. There are certain commands
which change the default behaviour of \texttt{GET} commands.

\texttt{IGNORE NEWLINE AFTER} \emph{controldefinestring}

\texttt{SETITEM SIZE} \emph{n} \texttt{PATTERN} $\{$
\emph{patterndefinestring }$\}$

\texttt{STOPGET IF SEE} \emph{controldefinestring}

The \texttt{IGNORE NEWLINE AFTER} command tells SMif which characters
or words at the end of a line have to be treated as a continuation
mark.  A continuation mark indicates that the line is to be treated as
continued in the next line (without line break). The \texttt{GET
  SINGLE}, \texttt{GET INT}, and \texttt{GET LIST} commands put
together two integer expressions before and after a continuation mark.
\texttt{GET POL} puts together two arbitrary words before and after a
continuation mark.

\texttt{SETITEM SIZE} is valid for \texttt{GET SINGLE}, \texttt{GET
  INT}, and \texttt{GET LIST} commands only. It is ignored for
\texttt{GET POL} commands. \texttt{SETITEM SIZE} defines how many
integer expressions will be treated together and what will happen with
these expressions.  We will see an example for this later. The
expressions are identified by the symbols \texttt{\$1}, \texttt{\$2},
...  \emph{patterndefinestring} is a mathematical formula over these
symbols.  For example,

\texttt{SETITEM SIZE 3 PATTERN \{\$1\}} \newline defines that only the
first expression of three expressions will be returned.  For
\texttt{GET INT}, this is the only valid type of
\emph{patterndefinestring}. The \texttt{GET SINGLE} and \texttt{GET
  LIST} commands are based on C long integers and it is possible to
apply any mathematical operation defined in the \emph{tcl} language to
the symbols \texttt{\$1}, \texttt{\$2}, ... In this case, valid
expressions are for example

\texttt{SETITEM SIZE 2 PATTERN \{int (cos(\$1)*tan(\$2))\}} \newline
or

\texttt{SETITEM SIZE 2 PATTERN \{\$1>\$2?\$1:\$2\}}.  \newline A
complete list of available operators can be found in the \texttt{expr}
man page of the \emph{tcl} library.

The \texttt{STOPGET} command tells SMif to stop moving the file
pointer as soon as a given character or the beginning of a given word
are read. This is one of the three conditions for a \texttt{GET}
command to terminate.  Furthermore, a \texttt{GET} command will
terminate if a \texttt{REPEAT} loop or a \texttt{GET} command
enclosing this \texttt{GET} command terminate, and of course, a
\texttt{GET} command will terminate at the end of the input file.

\subsubsection{The scope of control commands}

A \texttt{GET} command can contain a sub script. In particular, a
\texttt{GET LIST} command can contain another \texttt{GET LIST}
command.  In this way, SMif can deal with sub lists of lists.  The
validity domain of a definition (made by a \emph{controldefinestring})
is determined by the following rules.

1. The global level, i.\ e.\ the part of the SMif script not contained
in any \texttt{GET} command has level number 0.  Every \texttt{GET}
command increments the level number by 1 and all the definitions of
level number $n - 1$ are valid in level number $n$.

2. A definition by \texttt{IGNORE}, \texttt{IGNORE NEWLINE}, and
\texttt{STOPGET} remains valid until it is canceled by one of the
following \texttt{NOPOWER} commands on the same level:

\texttt{NOPOWER IGNORE} \emph{controldefinestring}

\texttt{NOPOWER IGNORE NEWLINE AFTER} \emph{controldefinestring}

\texttt{NOPOWER STOPGET} \emph{controldefinestring}

3. A valid \texttt{SETITEM} definition is canceled by a new
\texttt{SETITEM} command on the same level or by the command
\texttt{NOPOWER SETITEM} which restores the default behaviour.  The
default definition is \texttt{SIZE 1 PATTERN \{\$1\}}.

\textbf{Example.}

\begin{verbatim}
  # comment: example
  IGNORE {[WORD 123 234]}
  GET SINGLE {}
  GET SINGLE {
    NOPOWER IGNORE {[WORD] 123}
  }
  GET SINGLE {}
\end{verbatim}
In this example, the words \texttt{123} and \texttt{234} are ignored
by the first and the third \texttt{GET} command, while the second one
ignores only \texttt{234}. (The integer part in the word
\texttt{blabla1234} is not ignored since this is not a seperate
token.)

\subsubsection{The \texttt{SKIP} command}

While executing a \texttt{GET} command, the file pointer in the text
file is moved character by character. Apart from \texttt{GET} command,
\texttt{SKIP} is the only command which influences the file pointer
position:

\texttt{SKIP} \emph{n} \texttt{LINE}

\texttt{SKIP UNTIL SEE} \emph{controldefinestring}

\texttt{SKIP UNTIL SEEN} \emph{controldefinestring}

Suppose that the file pointer points at some position in line $x$.
The first \texttt{SKIP} command moves the file pointer to the
beginning of line $x + n$. $n$ can be positive or negative. This is
the only possibility to move the file pointer backwards in the text
file.

The other two commands move the text pointer forward until one of the
characters or words given by \emph{controldefinestring} are read (or
until an EOF condition is encountered).  \texttt{SKIP UNTIL SEE} moves
the pointer to the first character or the beginning of the first word
given in \emph{controldefinestring}.  \texttt{SKIP UNTIL SEEN} moves
the pointer to the position after the first specified character or
word. If a character or a word is specified by an \texttt{IGNORE}
command, this character or word will be ignored by every SMif command.
It is not possible to search for these strings.

\subsubsection{More \texttt{GET} commands}

The following variants for \texttt{GET} commands are defined:

\texttt{GET SINGLE \{} \emph{subscript} \texttt{\}}

\texttt{GET INT \{} \emph{subscript} \texttt{\}}

\texttt{GET LIST \{} \emph{subscript} \texttt{\}}

\texttt{GET LIST FOR NEXT} \emph{n} \texttt{ITEM \{} \emph{subscript}
\texttt{\}}

\texttt{GET LIST FOR NEXT} \emph{n} \texttt{LINE \{} \emph{subscript}
\texttt{\}}

\texttt{GET POL \{} \emph{subscript} \texttt{\}}

\texttt{GET POL FOR NEXT} \emph{n} \texttt{LINE \{} \emph{subscript}
\texttt{\}}

The \texttt{GET SINGLE} and \texttt{GET INT} commands return one
integer.  The first \texttt{GET LIST} command returns a list of single
precision integers and possible sublists while the \texttt{GET LIST
  FOR ... LINE} and \texttt{GET LIST FOR ... ITEM} return only a list
of single precision integers.

A list returned by a \texttt{GET LIST FOR NEXT} \emph{n} \texttt{ITEM}
command consists of exactly $n$ single precision integers.  A list
returned by a \texttt{GET LIST FOR NEXT} \emph{n} \texttt{LINE}
command consists of all the integers within the next \emph{n} lines
unless the \texttt{GET} command is terminated by a \texttt{STOPGET}
command or an enclosing \texttt{REPEAT} command or end of file is
reached.

\texttt{GET POL} gets a string and treats this string as a polynomial.
The command returns a \#--terminated polynomial in complete monomials.
This representation is readily understood by SIMATH functions. The
polynomial representation in the text file must suffice the following
conditions:

1. The coefficients of the complete monomials must be valid single
precision integers.

2. If some but not all multiplications of variables are denoted by
\texttt{$\star$}, this must not lead to ambiguities. For example,
\texttt{ab(a+b)} is a valid representation with variables \texttt{a}
and \texttt{b}. But this must not be written as
\texttt{ab$\star$(a+b)} since in this case SMif treats \texttt{ab} as
a new variable.

\emph{subscript} always denotes a sub script which is executed at the
beginning of a \texttt{GET} command. In the \emph{subscripts} of some
of the \texttt{GET} commands, not all of the control commands are
available. Sub scripts can be empty.

The following table shows which \texttt{GET} and control commands can
be used in sub scripts of a given \texttt{GET} command. Here, $+$
means that the operation is supported, and $-$ means that the
operation is not allowed.

\begin{displaymath}
\begin{array}{|c|c|c|c|c|c|c|c|}
\hline
                           & \texttt{GET}    & \texttt{GET} & \texttt{GET}  & \texttt{GET}  & \texttt{GET}  & \texttt{GET}  & \texttt{GET}  \\
                           & \texttt{SINGLE} & \texttt{INT} & \texttt{LIST} & \texttt{LIST} & \texttt{LIST} & \texttt{POL}  & \texttt{POL}  \\
                           &                 &              & \{ \}         & \texttt{ITEM} & \texttt{LINE} & \{ \}         & \texttt{LINE} \\
\hline
\texttt{IGNORE}            & +               & +            & +             & +             & +             & +             & +             \\
\hline
\texttt{IGNORE NEWLINE}    & +               & +            & +             & +             & +             & +             & +             \\
\hline
\texttt{SETITEM}           & +               & +            & +             & +             & +             & -             & -             \\
\hline
\texttt{STOPGET}           & +               & +            & +             & +             & +             & +             & +             \\
\hline
\texttt{SKIP}              & +               & +            & +             & +             & -             & +             & +             \\
\hline
\texttt{GET SINGLE (INT)}  & -               & -            & -             & -             & -             & -             & -             \\
\hline
\texttt{GET LIST}          & -               & -            & +             & -             & -             & -             & -             \\
\hline
\texttt{GET POL}           & -               & -            & -             & -             & -             & -             & -             \\
\hline
\end{array}
\end{displaymath}

As explained above, the use of \texttt{SETITEM} in \texttt{GET INT} is
restricted. Only the sub script of \texttt{GET LIST \{ ... \}} command
is allowed to contain further \texttt{GET LIST} commands. Apart from
this, no \texttt{GET} command is allowed to be contained in the sub
script of a \texttt{GET} command.

\subsubsection{Conditionals}

{~}

\texttt{REPEAT UNTIL SEE} \emph{controldefinestring} \texttt{\{}
\emph{script} \texttt{\}}

\texttt{IF SEE} \emph{controldefinestring} \texttt{\{} \emph{script}
\texttt{\}}

\texttt{IF SEE} \emph{controldefinestring} \texttt{\{} \emph{script1}
\texttt{\}} \texttt{ELSE} \texttt{\{} \emph{script2} \texttt{\}}

\texttt{SWITCH NEXTOBJ} \texttt{$\backslash$} \newline \hspace*{64pt}
\emph{controldefinestring\_1} \texttt{\{} \emph{script\_1} \texttt{\}}
\texttt{$\backslash$} \newline \hspace*{64pt} \texttt{.....}
\texttt{$\backslash$} \newline \hspace*{64pt}
\emph{controldefinestring\_n} \texttt{\{} \emph{script\_n} \textrm{\}}
\texttt{$\backslash$} \newline \hspace*{64pt} \texttt{OTHERWISE}
\texttt{\{} \emph{script\_n+1} \texttt{\}}

In SMif scripts, one can write a part of a command in a new line by
using the $\backslash$ symbol. (See the syntax of the \texttt{SWITCH
  NEXTOBJ} command above.)

The \texttt{REPEAT UNTIL SEE} command executes the sub script
\emph{script} until one of the following conditions is fulfilled.

1. A character or a word defined by \emph{controldefinestring} is
read.  \emph{controldefinestring} can be empty.

2. A \texttt{REPEAT} or \texttt{GET} command enclosing the
\texttt{REPEAT} command terminates.

3. End of file is reached.

An \texttt{IF SEE} command first tests whether the next character or
the next word is defined by \emph{controldefinestring}. In this case
\emph{script} resp.\ \emph{script1} is executed. If the condition is
not fulfilled and there is an \texttt{ELSE} part of the \texttt{IF
  SEE} command \emph{script2} is executed.

The \texttt{SWITCH NEXTOBJ} command works similar to the \texttt{IF
  SEE} command with the difference that there are arbitrary many cases
available. Only one sub script is executed.

Using the command described above one can write SMif scripts which
work on large classes of similar text files.

\textbf{Beispiel.} In most cases, SMif scripts will consist of a few
lines only. Here, we give an example of a very complicated script to
demonstrate the use of the different SMif commands. Suppose we want to
process the following text file:

\begin{verbatim}
  *** TEXT FILE: textf
  *** BLOCK a
      1 2 3 4 5 6 7 8
  *** BLOCK b
      1111 2\
      2\
      22 3333 4444
  *** BLOCK c
      123123123123123123
  *** BLOCK d
      %1 = 2x(x+y(2x+y))
  *** END
\end{verbatim}

We consider the follwing SMif script:

\begin{verbatim}
1  # comment line.
2  # scriptname: scriptf
3  
4  IGNORE NEWLINE AFTER {[CHAR] \\}
5  REPEAT UNTIL SEE {} {
6    SKIP UNTIL SEEN {[WORD] BLOCK}
7    SWITCH NEXTOBJ \
8      {[CHAR] a} { SETITEM SIZE 4 PATTERN {$1+$2+$3+$4}
9                   REPEAT UNTIL SEE {[CHAR] *} {
10                    GET SINGLE {}
11                  }
12                  NOPOWER SETITEM
13                } \
14     {[CHAR] b} { GET LIST {
15                    REPEAT UNTIL SEE {[CHAR] *} {
16                      GET LIST FOR NEXT 2 ITEM {}
17                    } 
18                    STOPGET IF SEE {[CHAR] *}
19                  }
20                } \
21     {[CHAR] c} { GET INT {} } \
22     {[CHAR] d} { SKIP UNTIL SEEN {[CHAR] = }
23                  GET POL {
24                    STOPGET IF SEE {[CHAR] *}
25                  }
26                }
27    # end SWITCH command
28 }
\end{verbatim}

\begin{itemize}
\item Line 1: Comments start with a \# sign.
\item Line 4: $\backslash$ characters must be escaped. Similar for
  \texttt{'\{'}, \texttt{'\}'}, \texttt{'\#'} characters.
\item Line 5: The subscript of the \texttt{REPEAT} command is repeated
  until end of file is reached.
\item Line 6: After the \texttt{SKIP UNTIL SEEN} command has been
  executed, the file pointer points at the first character after the
  word \texttt{BLOCK}.
\item Line 7: The next object is the character \texttt{'a'},
  \texttt{'b'}, \texttt{'c'}, \texttt{'d'}, or the word \texttt{"a"},
  \texttt{"b"}, \texttt{"c"}, \texttt{"d"}.
\item Line 8: Here \texttt{\{[CHAR] a\}} is the same as
  \texttt{\{[WORD] a\}}
\item Line 10: This \texttt{GET} command is repeated twice. The first
  time it returns $1+2+3+4 = 10$ and the second time it returns
  $5+6+7+8 = 26$.
\item Line 12: The \texttt{NOPOWER SETITEM} command cancels the
  definition of line 8.
\item Lines 14--19: This returns the list \texttt{( (1111 2222) (3333
    4444) )}.
\item Line 24: \texttt{GET POL} reads the polynomial
  \texttt{2x(x+y(2x+y))} and returns \texttt{2*x\^{}2*y\^{}0 +
    4*x\^{}2*y\^{}1 + 2*x\^{}1*y\^{}2\#}.
\end{itemize}

If a SIMATH program executes

\texttt{SMifRes\_FromFile(smifres, "textf", "scriptf")}; where
\texttt{smifres} is an initialized variable of type \texttt{SMifRes},
the following contents woll be assigned to the variable
\texttt{smifres}.

Number of entries: \texttt{SMifRes\_Count(smifres) = 5}

Entry 1: \texttt{SM\_single}: "10"

Entry 2: \texttt{SM\_single}: "26"

Entry 3: \texttt{SM\_list}: "( (1111 2222) (3333 4444) )"

Entry 4: \texttt{SM\_int}: "123123123123123123"

Entry 5: \texttt{SM\_pol}: "2*x\^{}2*y\^{}0 + 4*x\^{}2*y\^{}1 +
2*x\^{}1*y\^{}2\#"

\end{document}
